%*****************************************
\chapter{Ergebnisse}\label{ch:results}
%*****************************************

Im Folgenden werden die auf Grundlage der durchgeführten Interviews und der bisherigen Überlegungen erstellten Modelle und Diagramme vorgestellt. Dabei werden zunächst die einzelnen Interessengruppen aufgelistet, die in dem Umfeld der medizinischen Versorgung von Wohnungslosen Einfluss ausüben. Mithilfe von Kommunikationsdiagrammen der \ac{UML} werden einige konkrete Angebote oder Gegebenheiten näher beleuchtet und anhand des 3-Ebenen-Metamodells (\acs{3LGM²}) wird ein Gesamtbild des Leipziger Informationssystems für die medizinische Dokumentation beschrieben.

\section{Stakeholder-Analyse}

\begin{table}[ht]
	\centering
	\begin{tabulary}{\textwidth}{Lp{0.7\linewidth}}
		\toprule
		Stakeholder&				Interessen\\
		\midrule
		Wohnungslose&				Zugang zu medizinischer Versorgung\\
		&							Vermeidung von Diskriminierung\\
		&							evtl. Wahrung der Anonymität\\
		EU-Bürger und Drittstaatler&Zugang zu medizinischer Versorgung\\
		&							Wahrung der Anonymität\\
		Einwohner&					Keine Beeinträchtigung ihrer med. Versorgung\\
		CABL und UVO&				Verbesserung der med. Versorgung von Wohnungslosen\\
		&							Lückenlose Dokumentation\\
		Safe&						Verbesserung der Lebenssituation von Wohnungslosen\\
		&							standardisierte Vorgehensweise bei med. Vorfällen\\
		Sozialarbeiter:innen&		Beibehaltung der persönlichen Kontakte\\
		&							Belastungsreduzierung\\
		Ärzt:innen&					Verfügbare Dokumentation für Diagnose und Behandlung\\
		&							Belastungsreduzierung / kein wesentlicher Mehraufwand\\
		Sozialamt&					Erfüllung der Aufgaben zur Sozialhilfe\\
		&							Kostenminimierung\\
		Gesundheitsamt&				Erfüllung der Aufgaben zur Gesundheitsverwaltung\\
		&							Kostenminimierung\\
		\bottomrule
	\end{tabulary}
	\caption[Stakeholder]{Übersicht der Stakeholder}
	\label{tab:stakeholder}
\end{table}


\section{Kommunikationsmodelle}

\todo{Diagramme einfügen}

\subsection{CABL-Sprechstunden}

\subsection{UVO-Koordination}

\subsection{Krankenhausumfeld}

\subsection{VGP und SpDi}


\section{Modell des Informationssystems}


\section{Vergleich mit bestehenden Ansätzen}
