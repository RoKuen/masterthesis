\chapter{Interviewleitfäden}

\section{Ehrenamtliche Ärzt:innen}

\begin{itemize}
	\item Unter welchen Umständen wird dokumentiert?
	\begin{itemize}
		\item Gibt es ggf. Unterschiede zwischen den einzelnen Situationen? (auf der Straße, in Notunterkünften, …)
	\end{itemize}
	\item Erfolgt eine Feststellung der Identität?
	\begin{itemize}
		\item Wenn ja, mit welchen Mitteln? (Ausweis, Eigenaussage, …)
	\end{itemize}
	\item Kann auf bestehende Dokumentation zurückgegriffen werden?
	\item ­	Was wird alles dokumentiert?
	\begin{itemize}
		\item Welche Daten werden erfasst?
		\item Welche Dokumente werden erstellt?
	\end{itemize}
	\item Wer dokumentiert bzw. ist hierfür verantwortlich?
	\item Mit welchen Personen/Institutionen werden die Daten geteilt?
	\item In welcher Form erfolgt die Kommunikation? (Brief, E-Mail, Telefonat, …)
	\item {}[falls relevant] Fragen zum Einsatz der Kobo Tool Box:
	\begin{itemize}
		\item Wie wird das Tool gehandhabt?
		\item Welche Daten werden dabei erfasst?
		\item In welcher Form und Struktur werden diese abgelegt?
	\end{itemize}
	\item Sind weitere Tools bzw. Anwendungen im Einsatz?
	\begin{itemize}
		\item Wenn ja, welche?
		\item Wie werden diese eingesetzt?
	\end{itemize}
\end{itemize}

\newpage

\section{Mitarbeiter von CABL}

\begin{itemize}
	\item Unter welchen Umständen wird dokumentiert? (Einzelsprechstunden, auf der Straße, in Notunterkünften, …)
	\begin{itemize}
		\item Gibt es Unterschiede (Umfang oder Inhalt der Dokumentation) zwischen den einzelnen Situationen?
	\end{itemize}
	\item Wird unter Umständen die Identität der Personen erfasst?
	\item Was wird alles dokumentiert?
	\begin{itemize}
		\item Welche Daten werden erfasst?
		\item Welche Dokumente werden erstellt?
	\end{itemize}
	\item Wer dokumentiert bzw. ist hierfür verantwortlich?
	\item Mit welchen Personen/Institutionen werden die Daten geteilt?
	\item In welcher Form erfolgt die Kommunikation (Brief, E-Mail, Telefonat, …)
	\begin{itemize}
		\item mit den genannten Personen/Institutionen?
		\item mit dem Sozial- bzw. Gesundheitsamt?
	\end{itemize}
	\item Fragen zum anonymen Behandlungsschein:
	\begin{itemize}
		\item Wie erfolgt die Ausstellung?
		\begin{itemize}
			\item Welche Voraussetzungen müssen erfüllt sein?
			\item Gibt es auch eine digitale Variante?
		\end{itemize}
		\item Wie wird der ABS verwendet bzw. wie wird damit umgegangen?
		\item Wie wird die Behandlung finanziert?
		\item Wie ist die Zusammensetzung der Gruppe, die den ABS benutzen? (Obdachlose, LGBTQ, ...)
	\end{itemize}	
\end{itemize}

\newpage

\section{VGP und sozialpsychiatrischer Dienst}

\begin{itemize}
	\item Über welche Angebote des VGP wird ein Kontakt zu Wohnungslosen bzw. Menschen ohne Krankenversicherung hergestellt?
	\item Unter welchen Umständen wird dokumentiert?
	\begin{itemize}
		\item Gibt es Unterschiede (Umfang oder Inhalt der Dokumentation) zwischen den einzelne Situationen?
		\item Wie oft ist das mobile Kontakt- und Beratungsteam unterwegs? Wer ist dabei?
	\end{itemize}
	\item Was wird alles dokumentiert?
	\begin{itemize}
		\item Erfolgt eine Feststellung der Identität? Wenn ja, wie?
		\item Welche Daten werden erfasst?
		\item Welche Dokumente werden erstellt?
		\item Wer dokumentiert bzw. ist hierfür verantwortlich?
	\end{itemize}
	\item Welche Tools / Anwendungen werden genutzt?
	\item Mit welchen Personen / Institutionen werden die Daten geteilt?
	\item In welcher Form erfolgt die Kommunikation?
\end{itemize}

\newpage

\section{Stationäre und ambulante Behandlung}

\begin{itemize}
	\item Gibt es konkrete Vorgaben zum Umgang mit wohnungslosen Menschen?
	\item Welche Tools/Anwendungen werden zur Dokumentation eingesetzt?
	\item Gibt es im Vergleich zum Standardvorgehen Unterschiede bei der Dokumentation, wenn
	\begin{itemize}
		\item der Patient oder die Patientin seine/ihre Identität nicht nachweisen will oder kann?
		\item kein Krankenversicherungsnachweis vorliegt?
		\item ein anonymer Behandlungsschein vorliegt?
		\begin{itemize}
			\item Wie wird mit diesem umgegangen?
		\end{itemize}
	\end{itemize}
	\item Über welchen Weg werden Patient:innen an Sie verwiesen?
	\begin{itemize}
		\item Besteht Kontakt zu Hilfsorganisationen für Wohnungslose? (z.B. CABL, Safe, TiMMi ToHelp, …)
		\begin{itemize}
			\item Wenn ja, in welcher Form findet der Austausch statt? (Brief, E-Mail, Telefonat, …)
		\end{itemize}
	\end{itemize}
	\item Wie werden Termine zur Weiterbehandlung organisiert?
	\begin{itemize}
		\item Wie werden die Betroffenen informiert bzw. kontaktiert?
		\item Wie ist das Vorgehen beim Fernbleiben von Behandlungsterminen?
	\end{itemize}
\end{itemize}

\newpage

\section{Sozialarbeiter:innen}

\begin{itemize}
	\item Werden medizinische Daten dokumentiert?
	\begin{itemize}
		\item Wenn ja, welche Dokumente und in welcher Form?
		\item Unter welchen Umständen?
	\end{itemize}
	\item Wie ist das Vorgehen bei medizinischen Vorfällen, wenn kein Arzt oder keine Ärztin vor Ort ist?
	\begin{itemize}
		\item Wie oft ist med. Personal bei Einsätzen (z.B. bei dem Hilfebus) mit dabei?
	\end{itemize}
	\item An wen bzw. welche Institution werden Wohnungslose mit gesundheitlichen Beschwerden verwiesen?
	\item Gibt es (Beratungs-) Angebote, um hilfsbedürftige Menschen wieder in die Regelversorgung zu überführen?
	\begin{itemize}
		\item Wenn ja, wie erfolgreich zeichnen sich diese aus?
	\end{itemize}
	\item Besteht außerhalb der konkreten Fälle regelmäßiger Kontakt zu med. Personal?
	\begin{itemize}
		\item Wenn ja, in welcher Form? (Brief, E-Mail, Telefonat, …)
	\end{itemize}
\end{itemize}