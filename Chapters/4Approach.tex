%*****************************************
\chapter{Lösungsansatz}\label{ch:approach}
%*****************************************

In Vorbereitung auf den praktischen Teil dieser Arbeit müssen noch einige Überlegungen gemacht werden. Neben der Definition von Vergleichskriterien sollen dazu noch die konkreten Leitfäden für die Interviews erstellt und entsprechende Modellierungsmethoden bzw. Diagrammtypen ausgewählt werden.

\section{Vergleichskriterien}

Im Folgenden wird eine Auswahl an Kriterien definiert, die eine Vergleichbarkeit zwischen bestehenden Ansätzen und dem Leipziger System ermöglichen soll.

\begin{enumerate}
	\item Funktionalität
	\begin{itemize}
		\item[] Zu welchem Grad erfüllt das System seine zugewiesene Funktion?
	\end{itemize}
	\item Zeitaufwand
	\begin{itemize}
		\item[] Wie viel Zeit wird bei der Bedienung in Anspruch genommen?
	\end{itemize}
	\item Ressourcenaufwand
	\begin{itemize}
		\item[] Welche zusätzlichen Ressourcen werden zur alltäglichen Bedienung benötigt? Hierzu zählt neben Hardwareanforderungen oder papierbasierten Dokumenten auch der jeweilige Personalaufwand.
	\end{itemize}
	\item Anpassbarkeit
	\begin{itemize}
		\item[] Wie schnell und genau können mit der Zeit aufkommende Anforderungen (z.B. gesetzlicher Natur) bedarfsgerecht umgesetzt werden?
	\end{itemize}
	\item Bedienbarkeit
	\begin{itemize}
		\item[] Wie einfach oder umständlich ist die Verwendung des Systems? Hierbei sollte auch das subjektive Empfinden der Anwender mit in die Bewertung einfließen.
	\end{itemize}
	\item Digitalisierungsgrad
	\begin{itemize}
		\item[] Zu welchem Anteil können die Systeme in digitaler Form betrieben werden? Dieser Aspekt ist dahingehend von Bedeutung, da die Verwendung digitaler Lösungen immense Vorteile in Bezug auf Verfügbarkeit der jeweiligen Inhalte bringen kann.
	\end{itemize}
\end{enumerate}


\section{Interviewleitfäden}

Als Methode zur Informationsbeschaffung werden hauptsächlich Interviews geführt. Wie zuvor beschrieben sollen gesammelte Informationen nicht nur detailliert, sondern auch vergleichbar sein. Aus diesem Grund sind halbstrukturierte bzw. Leitfaden-Interviews hierbei von besonderem Interesse, wobei in Vorbereitung auf die eigentliche Durchführung zunächst die jeweiligen Interviewleitfäden erarbeitet werden müssen.

Die relevanten Personen, die für eine Befragung in Frage kommen, lassen sich dabei zunächst in zwei Gruppen untergliedern: Zum einen sollten Menschen mit medizinischem Bezug gefragt werden, wie z.B. die Ärzte von CABL oder sonstiges med. Personal im Kontakt mit Wohnungslosen. Zum anderen können auch Sozialarbeiter:innen der verschiedenen Hilfsorganisationen einen Einblick in die Routinen und Verfahrensweisen bei gesundheitlichen Anliegen von Hilfesuchenden geben.

Für die \textbf{Mediziner} werden die folgenden Fragen in Bezug auf die medizinische Versorgung von (wohnungslosen) Menschen gestellt, die keinen Zugang zum Regelsystem bzw. keine Krankenversicherung besitzen:

\begin{itemize}
	\item ­	Was wird alles dokumentiert
	\begin{itemize}
		\item Welche Daten werden erfasst?
		\item Welche Dokumente werden erstellt?
	\end{itemize}
	\item Wer dokumentiert?
	\item Mit welchen Personen/Institutionen werden die Daten geteilt?
	\item In welcher Form erfolgt die Kommunikation? (Brief, E-Mail, Telefonat, …)
	\item Fragen zum anonymen Behandlungsschein:
	\begin{itemize}
		\item Wie erfolgt die Ausstellung?
		\begin{itemize}
			\item Welche Voraussetzungen müssen erfüllt sein?
			\item Gibt es auch eine digitale Variante?
		\end{itemize}
		\item Wie wird der ABS verwendet bzw. wie wird damit umgegangen?
		\item Wie wird die Behandlung finanziert?
	\end{itemize}
\end{itemize}

Die Fragen für die \textbf{Sozialarbeiter:innen} haben den gleichen Bezug wie die der Mediziner, jedoch steht hier mehr die Kommunikation mit den Einrichtungen des Gesundheitswesens im Vordergrund. Idealerweise sollte die Befragung dieser Personengruppe nach den Interviews mit dem medizinischen Personal geschehen. Da sich die einzelnen Hilfsangebote einer breiten Auswahl an Aufgaben und Ziele annehmen, sollte man sich hier explizit auf die medizinische Versorgung beschränken:

\begin{itemize}
	\item Werden medizinische Daten dokumentiert?
	\begin{itemize}
		\item Wenn ja, welche Dokumente und in welcher Form?
	\end{itemize}
	\item Wie ist das Vorgehen bei medizinischen Vorfällen, wenn kein Arzt oder keine Ärztin vor Ort ist?
	\item An wen bzw. welche Institution werden Wohnungslose mit gesundheitlichen Beschwerden verwiesen?
	\item Besteht außerhalb der konkreten Fälle regelmäßiger Kontakt zu med. Personal?
	\begin{itemize}
		\item Wenn ja, in welcher Form? (Brief, E-Mail, Telefonat, …)
	\end{itemize}
\end{itemize}

Die Interviews werden soweit von den Teilnehmern nicht anders gewünscht in persönlicher Weise / vor Ort einzeln durchgeführt. Dies erleichtert zum einen die Wahl eines möglichen Zeitraumes, da nicht auf terminliche Verpflichtungen von mehreren Personen geachtet werden muss. Zum anderen ist anzunehmen, dass das persönliche Aufsuchen der Befragten mit einer geringeren Störung ihres Arbeitsalltages verbunden ist.

Zusätzlich zu den bereits aufgeführten Fragen, kann auch um Zustimmung gebeten werden, eine Audio-Aufnahme vom Interview anzufertigen, um zu einem späteren Zeitpunkt noch einmal auf das Gesagte zurückgreifen zu können. Eventuell ist es auch möglich, dass Modelle bzw. Diagramme, die auf dieser Grundlage erstellt werden, von den befragten Personen auf Korrektheit und Verständlichkeit überprüft werden.


\section{Überlegungen zur Modellierung}

\todo{Informationsfluss-Diagramm, Kommunikationsdiagramm, Stakeholderanalyse, 3LGM²}
