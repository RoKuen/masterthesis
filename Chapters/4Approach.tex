%*****************************************
\chapter{Lösungsansatz}\label{ch:approach}
%*****************************************

In Vorbereitung auf den praktischen Teil dieser Arbeit müssen noch einige Überlegungen gemacht werden. Neben der Definition von Vergleichskriterien sollen dazu noch die konkreten Leitfäden für die Interviews erstellt und entsprechende Modellierungsmethoden bzw. Diagrammtypen ausgewählt werden.

\section{Vergleichskriterien}

Im Folgenden wird eine Auswahl an Kriterien definiert, die eine Vergleichbarkeit zwischen bestehenden Ansätzen und dem Leipziger System ermöglichen soll. Dabei muss zunächst unterschieden werden, für welche Anwendungsfälle die einzelne Kriterienmenge gedacht ist. Ein späterer Vergleich kann beispielsweise auf der Ebene einer konkreten Software oder des umliegenden Informationssystems geschehen, welches alle Bausteine dieses soziotechnischen Systems mit einschließt.

Für den Vergleich von Informationssystemen kann auf \citet{Winter.2023} zurückgegriffen werden, wobei man hier die einzelnen Kriterien anhand von drei Ebenen unterteilt, welche auch den Grundbaustein für das 3LGM²-Metamodell darstellen:

In der \textbf{fachlichen Ebene} geht es darum, dass die richtigen Informationen zur richtigen Zeit am richtigen Ort den richtigen Personen in der richtigen Form zur Verfügung stehen, damit diese die richtigen Entscheidungen treffen können. Weiterhin sind in Bezug auf Datenqualität die Integrität, Authentizität und Korrektheit der Daten von Bedeutung, welche in möglichst standardisierter Form erfasst werden sollten.

Da in der \textbf{logischen Werkzeugebene} mit Anwendungssystemen als Komponenten gearbeitet wird, gibt es hier einen gewissen Grad an Überschneidung mit der oben aufgeführten Qualität von Softwareprodukten, da diese durch Customizing letztendlich als ein solches System angesehen werden. Aus diesem Grund werden auch hier Kriterien wie Funktionalität, Portabilität bzw. Anpassbarkeit, Zuverlässigkeit und Nutzerfreundlichkeit aufgeführt, wobei zusätzlich auch Zertifizierung und Interoperabilität mit genannt werden.

Die Qualität auf der \textbf{physischen Werkzeugebene} wird vordergründig anhand von drei Aspekten untersucht. So ist nicht nur die Verfügbarkeit und die Sicherheit von datenverarbeitenden Systemen von Bedeutung, sondern auch deren multiple Verwendbarkeit.

Sollte ein Vergleich konkreter Software bzw. für Anwendungssysteme durchgeführt werden, so kann zur Bewertung der Software-Qualität nach der ISO/IEC 25000 die folgenden sechs Kriterien aufgezählt werden:

\begin{enumerate}
	\item Funktionalität
	\item Zuverlässigkeit
	\item Effizienz
	\item Wartbarkeit
	\item Portabilität
	\item Nutzbarkeit
\end{enumerate}

Es ist jedoch anzumerken, dass der Fokus auf dem Informationssystem als Ganzes liegt. Insofern sei die Betrachtung und der Vergleich einzelner Komponenten eher hintergründig und wird nur durchgeführt, wenn es einen konkreten Anlass dafür gibt, z.B. wenn Anwendungen oder Tools im Einsatz sind, die auf das Aufgabengebiet der medizinischen Dokumentation bei vulnerablen Gruppen zugeschnitten sind.


\section{Interviewleitfäden}

Als Methode zur Informationsbeschaffung werden hauptsächlich Interviews geführt. Wie zuvor beschrieben sollen gesammelte Informationen nicht nur detailliert, sondern auch vergleichbar sein. Aus diesem Grund sind halbstrukturierte bzw. Leitfaden-Interviews hierbei von besonderem Interesse, wobei in Vorbereitung auf die eigentliche Durchführung zunächst die jeweiligen Interviewleitfäden erarbeitet werden müssen. (siehe Anhang A)

Bei der Auswahl der zu befragenden Personen wäre zunächst das Umfeld von CABL interessant, was zum einen organisatorische Mitarbeiter:innen als auch die ehrenamtlich arbeitenden Ärzt:innen mit einschließt. Weiterhin sollte auch medizinisches Personal befragt werden, was in Krankenhausbereichen arbeitet, die von wohnungslosen Menschen regelmäßig aufgesucht werden (z.B. Psychotherapie). Zuletzt können auch die Sozialarbeiter:innen der verschiedenen Hilfsorganisationen einen Einblick in die Routinen und Verfahrensweisen bei gesundheitlichen Anliegen von Hilfesuchenden geben.

Damit das Themengebiet besser eingeordnet werden kann, wird für die \textbf{ehrenamtlichen Ärzt:innen} zunächst angemerkt, dass sich die aufgeführten Fragen auf die medizinische Versorgung von (wohnungslosen) Menschen beziehen, die keinen Zugang zum Regelsystem bzw. keine Krankenversicherung besitzen. Der Leitfaden hierbei wird mit den Gedanken erstellt, um herauszufinden, wie die einzelnen Ärzt:innen bei einer Behandlung vorgehen, welche Tools sie dabei benutzen und ob sie eigene Notizen erstellen.

Die Fragen für die \textbf{Mitarbeiter von CABL} ähneln inhaltlich dem vorherigen Fragebogen, wobei die Fragen zur Dokumentation etwas generischer sind. Es vor allem herausgefunden werden, ob der Verein selber Dokumente zur medizinischen Versorgung anlegt oder erhält und wie gegebenenfalls damit umgegangen wird. Neben den Verfahrensweisen mit dem ABS wäre es auch wissenswert, wie der Kontakt zu anderen Institutionen im Gesundheitswesen oder dem Sozial- bzw. Gesundheitsamt aussieht.

Bei der Befragung des \textbf{medizinischen Krankenhauspersonals} steht im Vordergrund, wie mit der oben genannten Personengruppe umgegangen wird und welche Dokumente zur Behandlung benötigt werden. Als Teil der Regulärversorgung muss dahingehend auch untersucht werden, über welche Schnittstellen mit den Hilfsorganisationen kommuniziert wird und wie der Kontakt mit den wohnungslosen Personen aufgebaut wird und erhalten bleibt.

Die Fragen für die \textbf{Sozialarbeiter:innen} haben den gleichen Bezug wie die der Mediziner, jedoch steht hier mehr die Kommunikation mit den Einrichtungen des Gesundheitswesens im Vordergrund. Idealerweise sollte die Befragung dieser Personengruppe nach den Interviews mit dem medizinischen Personal geschehen. Da sich die einzelnen Hilfsangebote einer breiten Auswahl an Aufgaben und Ziele annehmen, sollte man sich hier explizit auf die medizinische Versorgung beschränken.

Die Interviews werden sofern von den Teilnehmern nicht anders gewünscht einzeln und in persönlicher Weise / vor Ort durchgeführt. Dies erleichtert zum einen die Wahl eines möglichen Zeitraumes, da nicht auf terminliche Verpflichtungen von mehreren Personen geachtet werden muss. Zum anderen ist anzunehmen, dass das persönliche Aufsuchen der Befragten mit einer geringeren Störung ihres Arbeitsalltages verbunden ist.

Zusätzlich zu den bereits aufgeführten Fragen, kann auch um Zustimmung gebeten werden, eine Audio-Aufnahme vom Interview anzufertigen, um zu einem späteren Zeitpunkt noch einmal auf das Gesagte zurückgreifen zu können bzw. dieses zu transkribieren. Eventuell ist es auch möglich, dass Modelle bzw. Diagramme, die auf dieser Grundlage erstellt werden, von den befragten Personen auf Korrektheit und Verständlichkeit überprüft werden.


\section{Überlegungen zur Modellierung}

Um gesammelte Informationen und Überlegungen anschaulich darzustellen, muss vorab eine Entscheidung über die zu verwendeten Modelle und Diagramme getroffen werden. Auf diese Weise sind benötigte Informationen klar aufgelistet und Fragen in den Interviews können daraufhin angepasst werden.

Als erstes würde sich zur besseren Beleuchtung des Umfelds eine Stakeholder-Analyse anbieten. Hierbei werden sämtliche Interessensparteien gesucht sowie deren Erwartungen und Einstellungen zu dem Sachverhalt näher analysiert. Dies sollte sich anschließend auch in den Anforderungen an ein entsprechendes Dokumentationssystem widerspiegeln, sodass die einzelnen Interessen weitestgehend abgedeckt sind.

Das zu untersuchende System soll nicht nur auf seine Struktur, sondern auch auf Veränderungen über Zeit untersucht werden. Aus den Verhaltensdiagrammen der UML bieten sich hierfür das Informationsflussdiagramm und das Kommunikationsdiagramm als mögliche Darstellungsformen an. Ersteres stellt den Informationsaustausch zwischen den einzelnen Entitäten in abstrahierter Form dar und eignet sich somit als genereller Überblick über das untersuchte Umfeld. Es ist jedoch in seiner Ausdruckskraft ziemlich limitiert, da keine konkreten Details über die übermittelten Informationen erbracht werden.

Ein Kommunikationsdiagramm ähnelt zunächst einem Sequenzendiagramm in der Hinsicht, dass es ebenfalls die Interaktionen zwischen Objekten oder Lebenslinien (engl. \textit{lifelines}) darstellt. Es legt den Fokus jedoch nicht auf das zeitliche Geschehen bzw. die Abfolge der einzelnen Interaktionen, sondern wird hierbei besser aufgezeigt, welche Elemente miteinander kommunizieren. Da in dieser Arbeit mehr Wert auf die einzelnen Akteure sowie deren Kommunikation untereinander gelegt wird, ist dieses Diagramm dem Sequenzendiagramm vorzuziehen.

Zur Visualisierung von Prozessen kann als Modell und Notation BPMN herangezogen werden. Dies ist von Relevanz, sollte ein konkreter Ablauf innerhalb des Informationssystems näher untersucht werden, wobei beispielsweise der Umgang mit dem ABS interessant erscheint. Weiterhin könnte man damit aufzeigen, wann in einem Prozess bestimmte Daten oder Dokumente erstellt bzw. diese benötigt werden. Um die Anzahl der angestrebten Modellarten zu begrenzen, wird in dieser Arbeit jedoch auf BPMN verzichtet, zumal es sich in seiner Aussagekraft teilweise mit dem Kommunikationsdiagramm überschneidet.

Zuletzt kann das System in seiner Gesamtheit mithilfe des 3LGM² Metamodells abgebildet werden. Anhand der fachlichen, logischen und physischen Ebene werden nicht nur die Beziehungen zwischen Aufgaben, Objekttypen und Organisationseinheiten dargestellt, sondern auch die darunterliegenden Anwendungsbausteine sowie letztendlich die physischen Datenverarbeitungsbausteine aufgezeigt. Da in dieser Arbeit das Informationssystem der medizinischen Dokumentation bei Wohnungslosen als Ganzes untersucht und aufgezeigt werden soll, biete dieses Metamodell die ideale Grundlage, um ein solches System zu visualisieren.

\todo{evtl. Bilder einfügen}