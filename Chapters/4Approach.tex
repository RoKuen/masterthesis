%*****************************************
\chapter{Lösungsansatz}\label{ch:approach}
%*****************************************

In Vorbereitung auf den praktischen Teil dieser Arbeit müssen noch einige Überlegungen gemacht werden. Neben der Definition von Vergleichskriterien sollen dazu noch die konkreten Leitfäden für die Interviews erstellt und entsprechende Modellierungsmethoden bzw. Diagrammtypen ausgewählt werden.

\section{Vergleichskriterien}

Im Folgenden wird eine Auswahl an Kriterien definiert, die eine Vergleichbarkeit zwischen bestehenden Ansätzen und dem Leipziger System ermöglichen soll.

\begin{enumerate}
	\item Funktionalität
	\item Zeitaufwand
	\item Ressourcenaufwand
	\item Anpassbarkeit
	\item Bedienbarkeit
\end{enumerate}
\todo{ergänzen und beschreiben}


\section{Interviewleitfäden}

\begin{itemize}
	\item Mit welchen Institutionen stehen Sie regelmäßig in Kontakt?
	\item Welche Dokumente bzw. Daten werden ausgetauscht?
	\item Auf welchem Weg erfolgt die Kommunikation?
\end{itemize}
\todo{ausbauen}

\section{Überlegungen zur Modellierung}

\todo{Kommunikations-diagramm, Stakeholder-Analyse, 3LGM²}