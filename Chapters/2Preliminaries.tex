%*****************************************
\chapter{Grundlagen}\label{ch:preliminaries}
%*****************************************

Um die genannten Ziele anzugehen, müssen zunächst die zugrundeliegenden Begriffe definiert und der zu untersuchende Bereich etwas näher beschrieben werden. Wie auch bei anderen Personengruppen in vulnerablen Situationen können auch die Lebenssituationen von Wohnungs- bzw. Obdachlosen verschieden vom bereits bekannten oder etablierten Alltagsgeschehen sein. Es ist demnach wichtig sich vor Augen zu führen, mit welchen Aufgaben oder Problemen sich diejenigen Personen konfrontiert sehen, die keinen festen Wohnsitz haben bzw. nur notdürftig untergekommen sind.

\section{Begriffserklärungen}

Der Begriff der Wohnungslosigkeit wird im Allgemeinen relativ weit gefasst und schließt eine verhältnismäßig große Gruppe von Personen ein. Die Lebenssituationen lassen sich jedoch im Hinblick auf die Vulnerabilität weiter abstufen, weshalb vom Verband FEANTSA - \citet{ethos} die Europäische Typologie der Wohnungslosigkeit (ETHOS) herausgegeben wurde. Diese definiert eine Wohnung mithilfe von drei Domänen: dem physischen Besitz, den sozialen Bereich für Privatheit und Beziehungen sowie einer rechtlichen Domäne mit legalem Rechtstitel. Darauf aufbauend werden vier Hauptkategorien von Lebenssituationen genannt:

\begin{enumerate}
	\item \textbf{Obdachlosigkeit}
	\begin{itemize}
		\item[] Dies schließt Menschen ein, die darauf angewiesen sind, auf öffentlichen Plätzen oder in Notschlafstellen und niederschwelligen Einrichtungen zu übernachten. Dies ist die Gruppe mit der höchsten Vulnerabilität, was auch meist mit einem begrenzten Ausschluss von der Gesellschaft verbunden wird.
	\end{itemize}
	\item \textbf{Wohnungslosigkeit}
		\begin{itemize}
			\item[] Personen in dieser Kategorie besitzen allgemein keinen festen Wohnsitz und kommen meist in Notunterkünften unter. Dies schließt neben Wohnungsloseneinrichtungen und Frauenhäusern auch Auffangstellen für Migranten und Asylbewerber ein, wobei die Aufenthaltsdauer meist kurzfristig begrenzt ist.
		\end{itemize}
		\newpage
	\item \textbf{Ungesichertes Wohnen}
		\begin{itemize}
			\item[] Menschen mit ungesicherten Wohnverhältnissen leben häufig bei Freunden, Bekannten oder Verwandten. Weiterhin werden hier auch Personen mit eingeschlossen, die von einer Delogierung/Zwangsräumung oder von häuslicher Gewalt bedroht werden.
		\end{itemize}
	\item \textbf{Ungenügendes Wohnen}
		\begin{itemize}
			\item[] Dies ist eine Zusammenfassung von physisch unzureichenden Wohnverhältnissen. Gemeint werden damit Behausungen, die nicht für konventionelles Wohnen geeignet sind wie z.B. Wohnwägen, Garagen oder Abbruchgebäude, sowie Wohnungen, in denen die Mindestquadratmeter pro Person unterschritten werden.
		\end{itemize}
\end{enumerate}

Es ist anzumerken, dass sich diese Typologie auf die konkreten Wohnsituationen der Betroffenen stützt, was nicht unbedingt mit weiteren Dimensionen wie z.B. Gesundheit oder Arbeitsverhältnis in Verbindung stehen muss. Im Folgenden wird weiterhin von Wohnungs- und Odachlosigkeit gesprochen, es soll jedoch alle Menschen mit einschließen, die sich in solchen vulnerablen Situationen befinden und aus verschiedenen Gründen ihre Identität nicht eindeutig ausweisen können oder wollen.

\section{Lebenssituationen von Wohnungslosen}

\section{Hilfsangebote}

\section{Länderspezifische Situationen}