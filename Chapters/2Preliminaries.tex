%*****************************************
\chapter{Grundlagen}\label{ch:preliminaries}
%*****************************************

Um die genannten Ziele anzugehen, müssen zunächst die zugrundeliegenden Begriffe definiert und der zu untersuchende Bereich etwas näher beschrieben werden. Wie auch bei anderen Personengruppen in vulnerablen Situationen können auch die Lebenssituationen von Wohnungs- bzw. Obdachlosen verschieden vom bereits bekannten oder etablierten Alltagsgeschehen sein. Es ist demnach wichtig sich vor Augen zu führen, mit welchen Aufgaben oder Problemen sich diejenigen Personen konfrontiert sehen, die keinen festen Wohnsitz haben bzw. nur notdürftig untergekommen sind.

\section{Begriffserklärungen}

Der Begriff der Wohnungslosigkeit wird im Allgemeinen relativ weit gefasst und schließt eine verhältnismäßig große Gruppe von Personen ein. Die Lebenssituationen lassen sich jedoch im Hinblick auf die Vulnerabilität weiter abstufen, weshalb vom Verband FEANTSA - \citet{ethos} die Europäische Typologie der Wohnungslosigkeit (ETHOS) herausgegeben wurde. Diese definiert eine Wohnung mithilfe von drei Domänen: dem physischen Besitz, den sozialen Bereich für Privatheit und Beziehungen sowie einer rechtlichen Domäne mit legalem Rechtstitel. Darauf aufbauend werden vier Hauptkategorien von Lebenssituationen genannt:

\begin{enumerate}
	\item \textbf{Obdachlosigkeit} (orig. \textit{roofless})
	\begin{itemize}
		\item[] Dies schließt Menschen ein, die darauf angewiesen sind, auf öffentlichen Plätzen oder in Notschlafstellen und niederschwelligen Einrichtungen zu übernachten. Dies ist die Gruppe mit der höchsten Vulnerabilität, was auch meist mit einem begrenzten Ausschluss von der Gesellschaft verbunden wird.
	\end{itemize}
	\item \textbf{Wohnungslosigkeit} (orig. \textit{houseless})
		\begin{itemize}
			\item[] Personen in dieser Kategorie besitzen allgemein keinen festen Wohnsitz und kommen meist in Notunterkünften unter. Dies schließt neben Wohnungsloseneinrichtungen und Frauenhäusern auch Auffangstellen für Migranten und Asylbewerber ein, wobei die Aufenthaltsdauer meist kurzfristig begrenzt ist.
		\end{itemize}
		\newpage
	\item \textbf{Ungesichertes Wohnen} (orig. \textit{insecure})
		\begin{itemize}
			\item[] Menschen mit ungesicherten Wohnverhältnissen leben häufig bei Freunden, Bekannten oder Verwandten. Weiterhin werden hier auch Personen mit eingeschlossen, die von einer Delogierung/Zwangsräumung oder von häuslicher Gewalt bedroht werden.
		\end{itemize}
	\item \textbf{Ungenügendes Wohnen} (orig. \textit{inadequate})
		\begin{itemize}
			\item[] Dies ist eine Zusammenfassung von physisch unzureichenden Wohnverhältnissen. Gemeint werden damit Behausungen, die nicht für konventionelles Wohnen geeignet sind wie z.B. Wohnwägen, Garagen oder Abbruchgebäude, sowie Wohnungen, in denen die Mindestquadratmeter pro Person unterschritten werden.
		\end{itemize}
\end{enumerate}

Es ist anzumerken, dass sich diese Typologie auf die konkreten Wohnsituationen der Betroffenen stützt, was nicht unbedingt mit weiteren Dimensionen wie z.B. Gesundheit oder Arbeitsverhältnis in Verbindung stehen muss. Im Folgenden wird weiterhin von Wohnungs- und Obdachlosigkeit gesprochen, es soll jedoch alle Menschen mit einschließen, die sich in solchen vulnerablen Situationen befinden und aus verschiedenen Gründen ihre Identität nicht eindeutig ausweisen können oder wollen.


\section{Lebenssituationen von Wohnungslosen}

Um die Umstände von Wohnungs- und Obdachlosen besser zu verstehen, kann es hilfreich sein, sich mittels statistischer Auswertungen einen Überblick über die Zusammensetzung dieser Menschengruppe zu verschaffen sowie die jeweiligen Gründe für das Leben ohne Wohnsitz zu erkunden. Die Daten basieren hierbei auf einen Statistikbericht der \ac{BAG W} zum Berichtsjahr 2020. Es muss zudem ein besonderes Augenmerk auf die gesundheitliche Situation gelegt werden, wobei weitere Publikationen herangezogen werden sollten.

\subsection{Allgemeine Daten}

Die Menge an Menschen in Wohnungslosigkeit deckt so gut wie alle Altersgruppen ab, wobei sich die Verteilung in den vergangenen Jahren nur wenig geändert hat. Anzumerken ist, dass die Gruppe der akut Wohnungslosen mit 73\% eine sehr große Teilmenge innerhalb aller Klient:innen im Dokumentationssystem für Wohnungslose ausmacht, weshalb die Unterschiede zwischen diesen beiden Gruppen sehr gering ausfallen können. Nichtsdestotrotz ist mit zu erkennen, dass akut Wohnungslose im Durchschnitt etwas jünger sind, als im gesamten Datensatz, welcher alle Personen mit einschließt, die Angebote der Arbeitsgemeinschaft unter Umständen präventiv mit in Anspruch genommen haben.

\begin{figure}[h]
	\centering
	\includegraphics[width=\linewidth]{Images/altersstruktur}
	\caption[Altersstruktur von Wohnungslosen]{Die Altersstrukur von Wohnungslosen nach Berichtsjahren \citep{BAGW.2022}}
	\label{fig:altersstruktur}
\end{figure}

Bei der Unterteilung nach biologischem Geschlecht ist auffällig, dass der Großteil (72,5\%) der erfassten Personen männlich ist. Innerhalb der akut Wohnungslosen ist der Anteil mit 76,1\% noch etwas höher. Dies liegt daran, dass Frauen vergleichsweise häufig \enquote{nur} unmittelbar von Wohnungslosigkeit bedroht werden oder in unzumutbaren Wohnverhältnissen leben. Dies lässt vermuten, dass sie eher dazu bereit sind, vorsorgliche und nachträgliche Hilfsangebote in Anspruch nehmen.

In Sachen Staatsangehörigkeit und Migration wurde über eine längere Zeit ein Zuwachs von Personen aus anderen Staaten beobachtet, wobei sich die Verhältnisse nach aktuellen Bekanntgaben stabilisiert haben. Der Anteil akut Wohnungsloser mit deutscher Staatsangehörigkeit beläuft sich auf 69,4\%. Den Rest stellen Personen aus anderen EU-Ländern (13,8\%), Menschen aus der nicht-EU (16,6\%) sowie eine kleine Gruppe, die als staatenlos erfasst wurden.

\subsection{Gründe für Wohnungslosigkeit}

Wie sich vermuten lässt, sind können sich die Gründe und Auslöser für Wohnungslosigkeit von Person zu Person stark unterscheiden. Es ist zudem möglich, dass ein Zusammenspiel von mehreren Aspekten zu solch einer drastischem Änderung der Lebenssituation führen kann. In Tabelle 2.1 sind die Gründe in einer Häufigkeitsverteilung angegeben, die 2020 von den betroffenen Personen als Hauptauslöser für eine drohende oder akute Wohnungslosigkeit mit angegeben wurden.

Bei der geschlechtlichen Unterscheidung sind in den meisten Punkten kaum bzw. nur geringe Unterschiede festzustellen. Lediglich bei Auslösern wie z.B. Haftantritt oder häuslicher Gewalt unterscheiden sich die Zahlen merkbar. Die Hauptauslöser bleiben weiterhin finanzieller oder zwischenmenschlicher Natur.

\begin{table}[ht]
	\centering
	\begin{tabulary}{\textwidth}{LRRR}
		\toprule
		Auslöser&							männl.&	weibl.&	\textbf{Gesamt}\\
		\midrule
		Miet- bzw. Energieschulden&			17,5\%&	18,6\%&	\textbf{17,8\%}\\
		Trennung/Scheidung&					15,7\%&	16,3\%&	\textbf{15,9\%}\\
		Ortswechsel&						16,3\%&	13,6\%&	\textbf{15,6\%}\\
		Konflikte im Wohnumfeld&			16,8\%&	16,0\%&	\textbf{16,6\%}\\
		Auszug aus der elterlichen Wohnung&	7,7\%&	8,5\%&	\textbf{7,9\%}\\
		Haftantritt&						8,8\%&	2,9\%&	\textbf{7,2\%}\\
		Arbeitsplatzverlust/-wechsel&		6,1\%&	3,3\%&	\textbf{5,4\%}\\
		Veränderung der Haushaltsstruktur&	3,9\%&	6,3\%&	\textbf{4,6\%}\\
		Krankheit&							2,4\%&	2,4\%&	\textbf{2,4\%}\\
		Gewalt durch Partner:in&			0,5\%&	7,3\%&	\textbf{2,4\%}\\
		Krankenhausaufenthalt&				1,4\%&	1,1\%&	\textbf{1,3\%}\\
		höhere Gewalt&						1,1\%&	1,1\%&	\textbf{1,1\%}\\
		Gewalt durch Dritte&				1,0\%&	1,8\%&	\textbf{1,2\%}\\
		institutionelle Nichthilfe&			0,8\%&	0,7\%&	\textbf{0,8\%}\\
		\bottomrule
	\end{tabulary}
	\caption[Auslöser für Wohnungslosigkeit]{Auslöser für Wohnungslosigkeit \citep{BAGW.2022}}
	\label{tab:gruende}
\end{table}

Laut \ac{BAG W} ist bei etwas über die Hälfte aller Fälle der Wohnungsverlust von Seitens der vermietenden Personen bzw. auf unfreiwillige Umstände zurückzuführen. Dies bedeutet jedoch nicht, dass der andere Teil freiwillig seinen festen Wohnsitz aufgibt. Viele kündigen einen bestehenden Betrag selbst oder ziehen ohne Kündigung aus, weil ihnen der Wohnungsverlust imminent erscheint, z.B. wenn Mietbeiträge nicht mehr tragbar sind.

\subsection{Bildung und Berufsleben}

In Hinsicht auf die Bildungssituation zeigt sich in diesem Umfeld im Vergleich zur Gesamtbevölkerung ein überproportionaler Anteil an Personen, die keine oder nur einen niedrig bewerteten Abschluss erreicht haben. Da laut \ac{BAG W} in den letzten Jahren ein geringer Anstieg an Hilfesuchenden mit höheren Bildungsqualifikationen oder sonstigen, z.T. nicht-deutschen Abschlüssen, zu verzeichnen ist, haben sich die Verhältnisse etwas in diese Richtung verschoben. Dennoch ist die Gruppe mit niedrigen Abschlüssen mit rund 66\% am meisten vertreten.

Eine weitere Erkenntnis lässt sich bei der Betrachtung der Berufsabschlüsse nach dem Alter (Abb. 2.2) herleiten: Der Anteil an Menschen ohne abgeschlossene berufliche Ausbildung nimmt mit dem Alter erwartungsgemäß ab. Da ältere Wohnungslose meist über einen praxis-, fachschul- oder (hoch-)fachschulbezogenen Berufsabschluss verfügen, lässt sich vermuten, dass generell das Alter oder eine eventuelle Altersarmut bedeutend zu Wohnungsverlusten beitragen könne.

\begin{figure}[h]
	\centering
	\includegraphics[width=\linewidth]{Images/berufsabschluss}
	\caption[Berufsabschluss nach Alter]{Berufsabschluss nach Altersklassen im Berichtsjahr 2020 \citep{BAGW.2022}}
	\label{fig:berufsabschluss}
\end{figure}

Ein weiterer Interessenspunkt ist die aktuelle Einkommenssituation bei akut Wohnungslosen. Dabei ist zu erkennen, dass über ein Drittel der Leute kein Einkommen haben und der überwiegende Teil auf Sozialleistungen angewiesen ist. Etwa jeder 10. Wohnungslose ist weiterhin berufstätig, was sich anteilig mit Beziehern von Rente oder Pension gleicht.

\subsection{Gesundheitliche Situation}

Im vorherigen Kapitel wurde erwähnt, dass bei wohnungslosen Menschen eine höhe Prävalenz von somatischen und psychischen Krankheiten beobachtet werden kann, was zu einer erhöhten Mortalität in dieser Kohorte beizutragen scheint. Diese ist im Vergleich zur Allgemeinbevölkerung zwei- bis fünffach erhöht.\citep[vgl.]{DAE228829} Zu den Todesursachen gehören neben Infektionskrankheiten aber auch Verletzungen, Vergiftungen und Suizide. \citep{Beijer.2011}

Trotz dieser Daten, haben Personen in akuter Wohnungs- und Obdachlosigkeit seltener Kontakt mit Ärzt:innen als andere, wobei ein verhältnismäßig großer Anteil im Zuge einer Notfallaufnahme oder über medizinische Hilfsangebote von Wohnungsnotfallhilfen geschieht. (Abb. 2.3)

\begin{figure}[h]
	\centering
	\includegraphics[width=\linewidth]{Images/aerztekontakt}
	\caption[Ärztekontakt nach Wohnungsnotfall]{Kontakt zu einem Arzt oder einer Ärztin in den letzten 6 Monaten nach Wohnungsnotfall \citep{BAGW.2022}}
	\label{fig:aerztekontakt}
\end{figure}

Diese Zurückhaltung in Bezug auf den Besuch von Institutionen im Gesundheitswesen kann zum einen mit den bereits erwähnten Gründen wie Angst vor Diskriminierung oder Scham \citep{Kaduszkiewicz.2017} erklärt werden. Es ist weiterhin nicht auszuschließen, dass insbesondere Obdachlose ihre Prioritäten anders setzen müssen. So wird z.B. die Suche nach einem Schlafplatz oder der Schutz des persönlichen Besitzes wichtiger erachtet als das eigentliche gesundheitliche Befinden, wodurch zugehörige Angebote nur in Notfällen angenommen werden.

Ein weiterer wichtiger Punkt ist für viele der eigene Krankenversicherungsstatus. Bei einer Befragung dazu, gaben nur 67\% der akut Wohnungslosen an, uneingeschränkten Zugang zu einer Krankenversicherung zu haben. Bei Obdachlosen liegt dieser Anteil sogar nur bei 59\%.

\section{Länderspezifische Situationen}

Zuerst sei gesagt, dass Wohnungs- und Obdachlosigkeit ein Problem ist, was eine Vielzahl von Ländern betrifft. Ein direkter Vergleich erscheint in vielen Fällen jedoch schwierig, da sich die verbreiteten Definitionen von dieser Art von Lebenssituation oftmals stark unterscheiden können. Während beispielsweise in Deutschland oder den nordischen Europaländern (Dänemark, Schweden, Finnland) fast alle Untergruppen der ETHOS-Light Typologie mit einbezogen werden, deckt sich im Süden und Osten Europas der Begriff der Wohnungslosigkeit oftmals mit dem der Obdachlosigkeit. Länder wie Großbritannien schließen sogar alle Menschen mit ein, denen der Wohnungsverlust erst bevorsteht oder derzeit in unzumutbaren Wohnverhältnissen leben. Obgleich der Unterschiede ist ETHOS ein hilfreiches Werkzeug, wenn erkennbar gemacht werden soll, welche Personengruppen in länderspezifische Statistiken mit einfließen. \citep{Busch-Geertsema.2018}

Auf Ebene der EU erweist sich Finnland als Land von Interesse. Als einziger Mitgliedsstaat ist hierbei schon jahrelang ein kontinuierlicher Rückgang der Wohnungslosenquote zu verzeichnen, da schon seit 1987 mit einer Vielzahl and nationalen Strategien versucht wird, (Langzeit-)Wohnungslosigkeit zu bekämpfen und zu reduzieren. \citet{Busch-Geertsema.2012} listet die europaweiten Trends der nationalen Strategien folgendermaßen auf:

\begin{itemize}
	\item Verstärkung und Optimierung von Prävention
	\item Beendigung von Wohnungslosigkeit bzw. bestimmter Formen als deutliches Ziel
	\item Vorgabe klarer Zielkriterien
	\item Reduzierung von Aufenthaltsdauern in Not- und Übergangsunterkünften
	\item Verbesserung von Unterkünften und Hilfsangeboten
\end{itemize}

In einem Artikel von \citet{Lopez.2022} wird in Bezug auf die USA auf eine Krise der Wohnungslosigkeit aufmerksam gemacht. Aufgrund von Inflation und stetig steigenden Mietpreisen, sind Wohnungen für viele US-Amerikaner finanziell nicht mehr tragbar, was bei Unterkünften zu erhöhtem Andrang führt. Neben den längeren Wartelisten tauchen in Parks und anderen öffentlichen Plätzen immer weitere Lagerplätze mit Zelten auf.


\section{Methoden zur Informationsbeschaffung}

Im Folgenden soll auf die verschiedenen Methoden eingegangen werden, die im Rahmen der Arbeit zum Zweck der Informationsbeschaffung von Bedeutung sind. Dies beinhaltet neben einer Beschreibung der zielgerichteten Literaturrecherche auch Grundlagen, die zur Findung eines Lösungsansatzes relevant sind.

\subsection{Literaturrecherche}

Um entsprechende Publikationen und Statistiken zu erhalten, wurden verschiedene Suchmethoden verwendet. Dabei sind neben einfachen Websuchmaschinen auch Literaturdatenbanken wie z.B. PubMed zum Einsatz gekommen, um erste Publikationen zu finden, die als Quelle in Frage kommen. Während Einträge in etablierten Datenbanken bereits weitestgehend als seriös eingestuft werden können, sollten Ergebnisse aus allgemeineren, internetweiten Suchalgorithmen dahingehend näher untersucht werden.

Als Schlüsselwörter, die bei der Suche verwendet wurden, sind neben den folgenden Wörtern auch verschiedene Kombinationen miteineinander sowie den Umständen entsprechend auch deren englischen Übersetzungen zu Einsatz gekommen:

\begin{itemize}
	\item Wohnungslose, Obdachlose, Wohnungs- und Obdachlosigkeit
	\item medizinische Dokumentation
	\item Hilfsangebote, Wohnungslosenhilfe in Leipzig
	\item Umgang mit vulnerablen Gruppen
	\item Ansätze, Strategien, Interventionen
\end{itemize}

Gefundene Dokumente wurden zudem als Anhaltspunkte genommen, um mithilfe des Schneeballverfahrens weitere mögliche Quellen ausfindig zu machen. Dies ist vor Allem hilfreich, da hierbei nicht nur verschiedene Studien zitiert werden, sondern auch auf Pressemitteilungen und Graue Literatur von verschiedenen relevanten Organisationen verwiesen wird.

Zuletzt konnten auch die Webseiten von Hilfsorganisationen, die auf einen der bereits genannten Wege gefunden wurden, auf weiterführende Links untersucht werden. Viele nutzen diese Möglichkeit, um anzugehende Probleme oder zu erreichende Ziele besser zu beschreiben oder zu untermalen.

\subsection{Interview}

Interviews werden für den praktischen Teil dieser Arbeit von besonders wichtigem Wert sein, da nach aktuellem Stand kaum Daten über das konkrete Informationssystem zur medizinischen Behandlung von Wohnungs- und Obdachlosen vorliegen oder bereits vorhandene Dokumente nicht vollständig genug sind. Damit neue Daten möglichst realitätstreu sind, ist es naheliegend eine weitreichende Auswahl an Akteuren im genannten Umfeld zu befragen.

Nach \citet{Doering.2015} bieten Interviews zahlreiche Vorteile gegenüber anderen Methoden. Zum Beispiel werden bei dieser Befragungsform subjektive bzw. nicht direkt beobachtbare Verhaltensweisen und Ereignisse zugänglich gemacht. Als Live-Situation können zudem bestimmte Hintergrundinformationen über die befragten Personen, wie Klarheit oder Zügigkeit der Antwort, mit aufgenommen werden, wobei die direkte Möglichkeit für eventuelle Rückfragen besteht. Auf diese Weise können in kurzer Zeit viele Informationen gesammelt werden, die sowohl quantitativ als auch qualitativ ausreichend sind.

Zu den Nachteilen zählen u.a. ein erhöhter Zeitaufwand bei der Vorbereitung, da neben dem konkreten Inhalt des Interviews die Befragungspersonen einzeln und persönlich kontaktiert werden müssen. Die geringere Anonymität als beispielsweise bei einer schriftlichen Befragung sowie die Reaktivität der Methode durch das generelle Wissen, dass man sich in einem Interview befindet, stellen außerdem mögliche Verzerrungsquellen dar.

Um den möglichen Problemquellen vorzubeugen, müssen somit zuvor einige Fragen beantwortet werden:
\begin{itemize}
	\item Wer soll befragt werden?
	\item Auf welche Weise erfolgt das Interview (persönlich, telefonisch, online)?
	\item In welcher Reihenfolge sollen die Kandidaten interviewt werden?
	\item Sind Gruppeninterviews möglich? Bringen sie einen Mehrwert?
	\item Wie wird strukturiert? Gibt es einen Leitfaden?
	\item Kann das Interview aufgezeichnet werden?
	\item Was sind mögliche Komplikationen?
\end{itemize}

Als Interviewform bietet sich ein sogenanntes halbstrukturiertes Interview an. Hierbei bedient man sich eines Leitfadens, damit die befragten Personen zwar die gleichen Fragen gestellt bekommen, diese aber dennoch aufgrund ihrer Offenheit mit eigenem Wortlaut beantworten können.

\subsection{weitere Methoden}

Neben einer allgemeinen Literaturrecherche und den Interviews gibt es noch einige weitere Methoden, die im Zuge der Arbeit von Bedeutung sein könnten. Beispielsweise wäre es vorstellbar, bestehende Vorgänge und Routinen mittels Beobachtungen zu untersuchen. Da hier nicht in die natürlichen Abläufe eingegriffen wird können neben einer realitätstreuen Darstellung der Prozesse auch unbewusste Verhaltensweisen der Beteiligten mit erfasst werden.

Sollte es erforderlich sein, eine Mehrzahl von Personen gleichzeitig zu bestimmten Punkten zu befragen, könnte auch ein Fragebogen ausgearbeitet werden. Je nach Grad der Standardisierung sind die Antworten untereinander vergleichbar und bieten die Möglichkeit, subjektive Meinungen und Unterschiede zu erfassen. Im Vergleich zu einem Interview ist ein Fragebogen zudem etwas diskreter und anonymer, sodass die Befragten eher dazu bereit sind, heikle Fragen zu beantworten.

Zuletzt kann bei bereits vorhandenen Dokumenten und Daten eine Dokumentenanalyse durchgeführt werden. Im Vergleich zu den anderen Methoden ist der Aufwand zwar geringer, jedoch kann nicht garantiert werden, dass relevante Informationen daraus gewonnen werden können, da die Dokumente, auf die zugegriffen wird, für andere Zwecke erstellt wurden. \citep{Doering.2015}