%*****************************************
\chapter{Ausführung der Lösung}\label{ch:solution}
%*****************************************

Nach der Erarbeitung des Lösungsansatzes soll es nun um die Einzelheiten gehen, die sich bei der Durchführung ergeben haben. Dazu steht primär die Abwicklung der Interviews sowie die Aufarbeitung der daraus entstehenden Rohdaten im Vordergrund. Es soll erklärt werden, welche Erkenntnisse aus den einzelnen Gesprächen gewonnen wurden.

\section{Durchführung der Interviews}

In der ursprünglichen Planung wurden insgesamt fünf Kontaktpersonen ausfindig gemacht, welche die bereits genannten Bereiche abdecken. Mit Fortschreiten der Gespräche wurden weitere Bereiche angesprochen, die in diesen Überlegungen von Relevanz sein könnten. Dies wären zum einen eine weitere Fachabteilung in einem Krankenhaus, die im höheren Maße mit Wohnungslosen bzw. Menschen, die den \ac{ABS} benutzen, in Kontakt kommt und zum anderen auch andere Streetworkprojekte (z.B. der Diakonie) sowie der sozialpsychiatrische Dienst, welcher dem Verbund Gemeindenahe Psychiatrie unterstellt ist und ebenfalls über ein mobiles Kontakt- und Beratungsteam verfügt. 

Schlussendlich wurden insgesamt sechs Interviews durchgeführt, die in den folgenden Unterkapiteln zusammengefasst werden. Sämtliche Gespräche wurden auf persönlicher Weise durchgeführt und audiotechnisch aufgezeichnet sowie anschließend transkribiert, sodass darauf basierende Modelle erstellt werden konnten.

\subsection{CABL und UVO}

Der Verein \ac{CABL} ist mit der Umsetzung des Projektes \ac{UVO} beauftragt, welches im Auftrag der Stadt Leipzig steht. An mehreren Anlaufstellen sollen ehrenamtlich arbeitende Ärzte Sprechstunden und Beratungsmöglichkeiten anbieten, um die niedrigeschwellige Versorgung für medizinische Hilfsangebote besser zugänglich zu machen. Die Ansätze untergliedern sich hierbei in einzelne Sprechstunden sowie aufsuchende Angebote.

Nach aktuellem Stand werden durch das Projekt wöchentliche Sprechstundenangebote in vier Übernachtungshäusern organisiert, davon eine Notunterbringung, eine Notschlafstelle für wohnungslose Frauen und zwei Zentren für Drogenhilfe. Weiterhin werden diese in ähnlicher Weise und ebenfalls nach Bedarf in zwei Leipziger Tagestreffs, namentlich der \enquote{Insel} und der \enquote{Oase}, angeboten, welche übergeordnet jeweils vom Suchtzentrum und gemeinsam von der Diakonie und dem Caritasverband getragen werden.

Bei den aufsuchenden Angeboten sind zwei Teams bestehend aus ehrenamtlich arbeitenden Ärzten jeweils mittwochs und sonntags mit privaten und nicht-privaten Fahrzeugen unterwegs und gehen aktiv auf betroffene Personen zu. Typische Standorte, die dabei aufgesucht werden, sind z.B. der Hauptbahnhof oder die Umgebung um Kirchen, also von Wohnungslosen stark besuchte Orte. Diese Angebote geschehen oftmals in Zusammenarbeit mit anderen Streetworkprojekten. So arbeitet das Sonntagsteam mit dem Hilfebus von \ac{Safe}, welcher wöchentlich am Sonntag am Hauptbahnhof hält.

Die Zeiten, in denen die Teams tätig sind, sind in der Regel auch den hilfesuchenden Personen bekannt, wodurch sich diese an den üblichen Orten einfinden, um anschließend eine medizinische Beratung oder eine akute Behandlung ihrer Beschwerden zu suchen.

Von seitens CABL bzw. der Projektkoordination werden hierbei keine verpflichtenden Vorgaben hinsichtlich der Dokumentation der Begegnungen und Behandlungen gemacht. Auf freiwilliger Basis kommt die sogenannte Kobo-Toolbox zum Einsatz: Den Ärzt:innen wird ein Weblink bereitgestellt, welcher zu einem webbasierten Formular führt, in dem fallbezogene Daten zu den Umständen erhoben werden können. Hierbei ist jedoch anzumerken, dass weder patientenbezogene Informationen noch andere Daten erfasst werden, die für eine weitere Behandlung benötigt werden.

Dieses Tool dient einzig dem Zweck der statistischen Auswertung und der Bestätigung gegenüber dem Sozial- bzw. Gesundheitsamt, dass Ziele des Projektes UVO aktiv angegangen werden. Die Formulare unterschieden sich demnach nur geringfügig zwischen den ehrenamtlichen Sprechstunden und der aufsuchenden Versorgung. Genauer gesagt, wird bei den Angeboten in Übernachtungshäusern und Tagestreff lediglich gefragt, ob die Person nur wegen dem Angebot gekommen ist, oder ob sie bereits an dem Ort ansässig ist.

\subsection{Ehrenamtliche Ärzt:innen}

Wie bereits erwähnt, ist dem ärztlichen Personal im Ehrenamt keine verpflichtende Dokumentation seitens UVO auferlegt. Es steht dementsprechend jedem Arzt und jeder Ärztin persönlich zu, in welchem Umfang er oder sie die Informationen über die behandelten Personen erfasst und welche Tools dabei eingesetzt werden. Bei den entsprechenden Interviews wurden einige Gründe mit angegeben, warum man sich patientenbezogene Daten mit notiert oder warum man dies unterlässt.

Im ersten Gespräch wurde im Bezug zu den aufsuchenden Angeboten erklärt, dass außer dem freiwilligen Ausfüllen des UVO-Formulars keine weiteren Notizen gemacht werden. Die Hauptbedenken hierbei waren größtenteils aus datenschutztechnischen Gründen, zumal man als Ärzt:in einer Verschwiegenheitserklärung unterliegt und somit Dokumente oder generell patientenbezogene Daten nicht ohne Weiteres mit anderen teilen kann.

Ein weiterer wichtiger Punkt, der auch von seitens UVO mit angemerkt wurde, ist, dass man schon froh darüber ist, dass manche Ärzte und Ärztinnen bereit sind, ehrenamtlich zu arbeiten und man aus diesem Grund deren Last nicht unnötig erhöhen möchte. Diese freiwillige Herangehensweise führt jedoch auch dazu, dass für die Sprechstunden in einem konkreten Übernachtungshaus gar keine Daten, beispielsweise zur Anzahl behandelter Patienten oder der durchgeführten Beratungen, vorliegen.

Wenn sich Ärzt:innen Notizen machen, dann dienen diese zumeist ausschließlich als persönliche Gedächtnisstütze und werden mit keiner weiteren Person oder Institution geteilt. Es werden in dieser Hinsicht auch keinerlei Dokumente, wie z.B. Arztbriefe oder Überweisungen erstellt, da sie hierbei nicht als Institution agieren.

Jegliche Kommunikation in diesem Umfeld erfolgt fallbezogen und zumeist auf persönlicher Ebene über Chatgruppen, telefonische Absprachen und E-Mail. Zu diesem Zweck wurden eine Vielzahl an Gruppen vorwiegend über die App \enquote{Signal} erstellt, jeweils mit unterschiedlichen Zusammensetzungen aus Ärzt:innen und Streetworker:innen sowie den dazugehörigen Teams.

Hierüber erfolgen Absprachen zur allgemeinen Organisation sowie zum erneuten Aufsuchen von behandelten oder beratenen Personen. Auf identifizierende Daten wie vollständiger Name oder Geburtsdatum wird weitestgehend verzichtet, häufig bedient man sich hier Spitznamen oder Pseudonymen, unter denen die jeweilige Person auf der Straße bekannt ist. Unter Zustimmung werden auch Foto-Aufnahmen von Wunden oder Ähnlichem erstellt, um über die genannten Gruppen eine Zweitmeinung einzuholen.

\subsection{Verbund Gemeindenahe Psychiatrie}

Der \ac{VGP} versteht sich als \enquote{ein teilstationäres und ambulantes Behandlungs- und Beratungszentrum für Menschen mit psychischen Erkrankungen oder	psychosozialen Problemen im Erwachsenenalter} \citep{VGP.2023} und behandelt bzw. berät Menschen unabhängig davon, ob sie krankenversichert sind. Um die Zugänglichkeit zu erhöhen und Sprachbarrieren abzubauen, besteht zudem die Möglichkeit, Dolmetscher zu bestellen, sodass auch mit Personen gearbeitet werden können, die kein Deutsch sprechen.

Es erfolgt eine Unterteilung in drei Betreuungsebenen: Die \ac{PIA} bietet an insgesamt fünf Standorten in Leipzig eine ambulante Betreuung und Behandlung für Menschen mit psychiatrischen Erkrankungen, in psychischen Stresssituationen oder mit einem komplexen und mehrschichtigen Hilfebedarf. Zum Leistungsspektrum gehören unter anderem die psychische und psychiatrische Diagnostik sowie Physio- und Ergotherapie.

Als Alternative zu einer vollstationären psychiatrischen Behandlung gibt es die \ac{TK} mit vier Standorten, an denen zusammen 60 Plätze in Anspruch genommen werden können. 7 davon sind der Gerontopsychiatrie zuzuordnen, welche sich mit der Gruppe älterer Klient:innen befasst. Zu den Aufnahmevoraussetzungen gehört neben einem Vorgespräch mit einem Psychologen auch ein gültiger Einweisungsschein, welcher durch einen entsprechenden Facharzt bzw. -ärztin, beispielsweise aus der Institusambulanz, oder Hausärzt:in ausgestellt werden kann.

Für diese Arbeit von höherem Interesse ist der Sozialpsychiatrische Dienst (\acs{SpDi}). Dieser ist als Pflichtaufgabe der Stadt Leipzig zu verstehen und ist gleich der \ac{PIA} an fünf Standorten aktiv, besitzt aber zusätzlich ein mobiles Kontakt- und Beratungsteam, welches ein aufsuchendes Angebot für wohnungs- bzw. obdachlose Menschen ist. Dieses Team ist generell täglich verfügbar, wird aber nur aktiv, wenn Personen gemeldet werden, z.B. durch Streetworker, Ordnungsamt oder Polizei, die psychische Hilfe benötigen. Die Zusammensetzung ist dahingehend variabel und kann aus Sozialarbeiter:innen, Ärzt:innen sowie begleitendes Personal bestehen.

\subsection{Krankenhauspersonal}

Im stationären und ambulanten Krankenhausumfeld sollen alle Personen unbeachtet ihrer Lebenssituation und Herkunft gleich behandelt werden. Für viele wohnungs- bzw. obdachlose Personen ist in diesem Zusammenhang die Notfallaufnahme der erste Kontaktpunkt im Krankenhaus, nachdem Passanten, Streeworker:innen oder Ärzt:innen der aufsuchenden Versorgung einen Krankenwagen gerufen haben.

Für die Personen wird dabei im Patientenverwaltungssystem jeweils ein neuer Fall angelegt, in dem die verfügbaren Daten eingetragen werden. Sollten die Informationen unvollständig sein, werden lediglich Anmerkungen verfasst und Pseudonyme verwendet. Auch ein fehlender Nachweis einer Krankenversicherung soll die Behandlung nicht beeinträchtigen, wobei die anfallenden Kosten vom Sozialamt getragen werden, sollte kein anderer Kostenträger gefunden werden.

Es ist zudem zu unterscheiden, auf welchem Weg die betroffenen Personen in die jeweilige Fachabteilung eingewiesen werden. Im Fall der Psychiatrie und Psychotherapie, werden die meisten über den Rettungsdienst vorstellig. Ein anderer Weg wäre, wenn sich jemand selbstständig meldet und sich um einen Einweisungstermin kümmert. Da hierbei jedoch ein konkreter Einweiser (z.B. Hausärzt:in) benötigt wird, wird bei den Fällen, wo diese Instanz fehlt, eine Notfallvorstellung durchgeführt.

\subsection{Sozialarbeiter:innen}

In Bezug auf Sozialarbeiter:innen und Streetworker:innen sei zunächst gesagt, dass diese das Ziel verfolgen, die allgemeine Situation von Wohnungslosen bzw. Menschen aus vulnerablen Gruppen soweit es geht zu verbessern. Die medizinische Versorgung ist daher nur ein Teilgebiet, welches hierbei angegangen wird.

Im Zuge ihrer Arbeit bauen sie ein persönliches und vertrauensvolles Verhältnis zu Personen auf, die sich von der Gesellschaft ausgeschlossen fühlen. Dies ist mit Bezug auf die in Kapitel \ref{sec:reqMedCare} genannten Anforderungen an den Umgang mit vulnerablen Gruppen auch von besonders Wichtigkeit, da die betroffenen Menschen sich sonst nur bei größter Dringlichkeit an die Hilfsangebote wenden, was schlussendlich die eigentlichen Ziele wieder untermauern würde.

Der Aufgabenbereich, den die Sozialarbeiter:innen beim Streetwork angehen, umfasst dabei allgemeine Unterstützung wie z.B. die Ausgabe von Essen, Kleidung oder Hygieneartikel, aber auch die Möglichkeit, die Personen auf dem Weg zu einem Amt oder zu einem Arzt / einer Ärztin zu begleiten.

Für das Projekt \ac{Safe} kommen insgesamt drei Teams zum Einsatz:
\begin{itemize}
	\item Team "Konsum"
	\begin{itemize}
		\item[] Aktiv im West Leipzigs versucht das Team Kontakte zu Wohnungslosen aufzubauen und zu erhalten. Des Weiteren steht es den Bedürftigen bei Problemen im Alltag oder z.B. bei Suchtfragen unterstützend zur Seite.
	\end{itemize}
	\item Team "Wohnen"
	\begin{itemize}
		\item[] Diese Team ist vorwiegend im Leipziger Norden unterwegs und hat die erweiterte Aufgabe, auch Menschen in ungeklärten oder schwierigen Wohnsituationen zu helfen.
	\end{itemize}
	\item Hilfebus
	\begin{itemize}
		\item[] Der Hilfebus des Suchtzentrums ist im gesamten Stadtgebiet aktiv und arbeitet eng mit den bisher genannten Teams sowie anderen Streetwork-Projekten z.B. von der Diakonie zusammen. An bestimmten Schwerpunkten hält der Bus auch länger und bietet allgemein eine mobile Grundversorgung und Verweisberatung.
	\end{itemize}
\end{itemize}

Bei den Teams erfolgt ähnlich wie bei den Angeboten des Projektes \ac{UVO} eine interne anonyme Evaluierung, welche über KoboToolbox erfolgt. Als Grund wurde dabei angeführt, dass eine quantitative Aufführung der erbrachten Leistungen und der angetroffenen Personen für die Geldgeber von enormer Wichtigkeit ist. Diese Evaluationsdaten werden mittlerweile in allen Erwachsenen-Streetworkprojekten erhoben und der daraus entstehende Basisdatensatz kann anschließend z.B. von Drogen- oder Suchtbeauftragten der Stadt Leipzig abgerufen werden, um verschiedene Berichte daraus anzufertigen.

Die Kommunikation zwischen den einzelnen Personen und zusammengestellten Teams findet wie bereits erwähnt auf verschiedener Weise statt. Vorrangig werden neben E-Mail und Telefonie auch Chat-Apps wie Signal oder Telegram eingesetzt, in denen für die unterschiedlichen Zusammensetzungen (Hilfebus mit Team "Wohnen", Hilfebus mit Streetwork der Diakonie, ...) Chatgruppen erstellt wurden. Weiterhin finden auch Austauschrunden in Person oder online statt.

Der Kontakt zu medizinischem Personal besteht außerhalb der einzelnen Gruppen nur vereinzelt. Sollte eine Person ein medizinisches Anliegen haben, so wird primär versucht, dieses akute Problem zu lösen, was von erster Hilfe über eine Begleitung zu einem ärztlichen Termin bis hin zum Rufen eines Rettungswagens reicht. Bei diesem Fokus auf die Notversorgung gehen chronische Beschwerden wie psychische Krankheiten, Bluthochdruck oder Diabetes sehr oft unter.

Ein weiteres Problem, was in diesem Zusammenhang auch auftreten kann, sind bestimmte Grenzfälle, bei denen die Personen nicht als Notfall eingestuft werden aber dennoch soweit verletzt oder krank sind, dass sie nicht ohne Weiteres in Notschlafstellen aufgenommen werden können. Diese werden aufgrund der geringen Dringlichkeit aus den Notaufnahmen entlassen, bräuchten aber dennoch einen Ort, wo sie über längere Zeit liegen und medizinisch bzw. pflegerisch versorgt werden können, wofür die Notschlafstellen jedoch nicht ausgerüstet sind. 


\section{Aufbereitung der Audio-Aufnahmen}

Durch die Interviews sind jeweils zwei Arten an Rohdaten entstanden: Zum einen gehören sämtliche Notizen dazu, die während des Interviews per Hand verfasst wurden. Zum anderen wurden mit Erlaubnis der Beteiligten Audioaufnahmen der Gespräche mithilfe eines Smartphones erstellt. Die handschriftlichen Notizen dienen dabei zunächst nur als Gedankenstütze und werden nach der Bearbeitungszeit durch die Transkripte der m4a-Dateien als Grundlage abgelöst.

Um das aufgenommene Gespräch in Textform zu bringen, gibt es mehrere Herangehensweisen, die sich in Aufwand und Ergebnisqualität unterscheiden. So stehen für diese Aufgabe Methoden der manuellen und automatischen Transkription sowie die Hilfe durch Schreibkräfte zur Auswahl, wobei letztere Möglichkeit allein schon aus Kostengründen wegfällt.

Eine manuelle Transkription kann im einfachsten Fall ohne zusätzliche Tools erfolgen, indem die Audioaufnahme abgespielt wird, um per Hand das Gesagte aufzuschreiben. Da das Sprechen und Schreiben eines bestimmten Inhalts mit unterschiedlicher Geschwindigkeit geschieht, muss die Wiedergabe oft pausiert bzw. zurückgespult werden. Allein das mehrmalige Abspielen einzelner Passagen führt dazu, dass für diese Methode ein signifikanter Zeitbedarf anfällt, der durch Korrekturlesen und dem Setzen der Zeitmarken weiter erhöht wird.

Um den Aufwand zu verringern, werden von verschiedenen Herstellern Tools angeboten, mit denen das manuelle Transkribieren vereinfacht werden soll. So können die Audiodateien in anpassbarer Geschwindigkeit wiedergegeben und Zeitmarken per Knopfdruck eingesetzt werden. Diese Anwendungen sind zumeist kostenfrei und können sowohl webbasiert über einen Browser als auch als eigenständiges Programm bedient werden.

Bei einer automatischen Transkription wird versucht, Gesprochenes via Spracherkennung bzw. Speech-to-Text in eine lesbare Form zu bringen. Die dabei verwendeten Sprachmodelle schaffen diese Umwandlung in relativ kurzer Zeit, sind aber anfällig für Dialekte oder unbekannte/unverständliche Worte, zumal nicht jedes Modell für die deutsche Sprache eingesetzt werden kann. Viele der Anbieter bieten diese Option als kostenpflichtige Alternative zur manuellen Transkription an, da der Rechenaufwand auf ihrer Seite hierbei um einiges höher ist.

Eine andere Möglichkeit wurde mit dem allgemeinen Sprachmodell \textbf{Whisper} von OpenAI entdeckt, welche unter der MIT-Lizenz frei zugänglich ist und mit Python unter Verwendung der Bibliothek PyTorch trainiert und getestet wurde. Es besitzt daneben noch weiter Abhängigkeiten von bestimmten Python-Pakete, wie z.B. tiktoken, mit dem Text in Tokens aufgeteilt werden kann, die anschließend von verschiedenen Modellen verwendet werden können.

Bei der Verwendung von Whisper stehen insgesamt fünf Modellgrößen zur Auswahl, die sich in ihrer Parameteranzahl und Geschwindigkeit unterscheiden. Die kleinste Option ist \enquote{tiny} mit 39 Millionen Parametern, welche in der Spracherkennung etwa 32-mal schneller ist als das größte Modell \enquote{large} mit 1550 Millionen Parametern. Die Wahl fiel letztendlich auf die Größe \enquote{medium}, da es einen guten Kompromiss zwischen Ressourcen- bzw. Zeitaufwand und Ergebnisqualität darstellt.

Die einzelnen Aufnahmen konnten auf diesem Weg \enquote{im Hintergrund} in Textform gebracht werden und wurden in verschiedenen Dateiformaten, beispielsweise für Untertitel (.vtt, .srt), gespeichert. Nach manueller Korrektur und Formatierung, stehen die Transkripte nun zur Verfügung.