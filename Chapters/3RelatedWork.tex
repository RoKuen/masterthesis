%*****************************************
\chapter{Stand der Forschung}\label{ch:relatedWork}
%*****************************************

Medizinische Versorgung kann wie bereits erwähnt für viele Menschen ein heikles Thema sein. Insbesondere für Personen in Wohnungslosigkeit kann der Gang zum Arzt schon aufgrund von negativen Erfahrungen viel Überwindung kosten. Entsprechend müssen etwaige Hilfsangebote auf die Situation und die konkreten Personengruppen angepasst sein, damit ihr Angebot auch in Anspruch genommen wird.

\section{Anforderungen an die medizinische Versorgung von vulnerablen Gruppen}

Der Umgang mit vulnerablen Personengruppen ist schon länger ein Thema von bedeutendem Interesse und es wurden dahingehend auch Möglichkeiten untersucht, um den Zugang zu medizinischen Angeboten so einfach wie möglich zu gestalten. Im Zuge von Behandlungen ist zwischenmenschlicher Kontakt unausweichlich, wodurch das Personal erwartungsgemäß eine wichtige Schlüsselrolle beim Patientenkontakt einnimmt.

In \citet{Hwang.2014} wurden in Bezug auf die Arbeit mit Wohnungslosen einige Schlüsselpunkte zusammengestellt, die bei Gesundheitsdienstleistern mit beachtet werden sollten:

\begin{itemize}
	\item \textbf{Interpersonelle Beziehungen}
	\item[] Beim Umgang mit Wohnungslosen sollte immer entsprechender Respekt gezeigt werden, um die Würde des Gegenüber nicht zu verletzen. Man sollte den Menschen mit Wärme und Fürsorge entgegentreten und versuchen, gegenseitiges Vertrauen aufzubauen. In diesem Sinne könnten auch Peer-Beratungen, also Beratungen durch Menschen in (ehemals) ähnlichen Situationen, von Bedeutung sein.
	\item \textbf{Gemeinschaftsmittel}
	\item[] Dienstleister müssen mit den Hilfsprogrammen und Ressourcen vertraut sein, die den Patienten zur Verfügung stehen, oder in enger Zusammenarbeit mit Personal sein, die entsprechenden Vorkenntnisse besitzen. \newpage
	\item \textbf{Klinische Versorgung}
	\item[] Angepasste Richtlinien sollen den Dienstleistern helfen, den physischen und mentalen Gesundheitsbedürfnissen von Menschen in Wohnungslosigkeit gerecht zu werden. Dabei sollen Systeme etabliert werden, die Nachuntersuchungen sicherstellen und eine beiderseitige Kommunikation zwischen Krankenhäusern und gemeindenahen Anbietern erleichtern. Weiterhin sollten sogenannte Outreach-Programme mit hinzugezogen werden, um die Menschen aktiv zu erreichen.
	\item \textbf{Interessenvertretung}
	\item[] Dienstanbieter im Gesundheitswesen können sich für die Durchführung und Erhaltung von Maßnahmen einsetzen, die Verbesserungen der gesundheitlichen Situation von Wohnungslosen vorsehen.
\end{itemize}


\section{Hilfsangebote in Leipzig}

Not und Konflikte können viele Formen annehmen, weswegen sich mit der Zeit eine große Anzahl an Organisationen und Verbänden gebildet hat, die es sich zur Aufgabe gemacht haben, diese weitverbreiteten Probleme anzugehen. Nachfolgend ist eine Auswahl der Hilfsangebote aufgelistet, die in der Stadt Leipzig eine Anlaufstelle für Wohnungs- und Obdachlose bieten.

\subsection{Leipziger OASE}

Wie auch in anderen Teilen Deutschlands ist die Diakonie als Wohlfahrtsorganisation und sozialer Dienst der evangelischen Kirchen auch in Leipzig vertreten und bietet nach dem Leitbild der Nächstenliebe zahlreiche Hilfsangebote für Menschen in Notsituationen an. Zusammen mit anderen Verbänden wie Caritas und AWO gehört sie zur Bundesarbeitsgemeinschaft der Freien Wohlfahrtspflege, welche sich mit zentralen Problemen wie Armut, Arbeit, Gesundheit und Pflege, sowie Migration und Integration befassen.

In Bezug auf die Wohnungslosenhilfe wird zusammen mit dem Caritasverband die \enquote{Leipziger Oase} angeboten, die neben ihrer Funktion als erste Anlaufstelle auch als Ergänzung zu Übernachtungshäusern angesehen wird. Nach eigenen Angaben kommen täglich zwischen 70 und 90 Personen, denen neben grundlegenden Angeboten wie Essensversorgung, Duschmöglichkeit und Wäscheservice auch Beratungen, Begleitungen und Vermittlungen in verschiedenen Formen zur Auswahl stehen. Seit November 2019 ist zudem das Team der Straßensozialarbeit tätig, welches aktiv auf Obdachlose zugeht und somit auch Menschen in entlegenen Gebieten erreichen kann. \citep{Diakonie}

\subsection{CABL}

Der Verein \enquote{Clearingstelle und Anonymer Behandlungsschein Leipzig} befasst sich damit, notdürftigen Personen den Zugang zum Gesundheitssystem und die Suche nach Kostenträgern zu erleichtern. Dies ist von besonderer Bedeutung, sollte kein Versicherungsschutz bestehen, da ärztliche Behandlungen vertraulich, kostenfrei und vor allem anonym organisiert werden können. Entstanden ist CABL aus einer Initiative von Medinetz Leipzig und wird seit 2019 zudem von der Stadt gefördert.

Als zentrales Angebot steht neben den Sozialberatungen der anonyme Behandlungsschein im Zentrum, und fungiert ähnlich wie eine Erklärung der Kostenübernahme. Als bestehender Lösungsansatz wird dieser nochmals im entsprechenden Kapitel beschrieben. \citep{CABL}

\todo{evtl. ausführlicher beschreiben}

\subsection{Safe}

\enquote{Straßensozialarbeit für Erwachsene} - abgekürzt Safe - ist ein Projekt des Suchtzentrums Leipzig. Ähnlich dem Streetwork-Projekt der Diakonie suchen die Mitarbeitenden öffentliche Plätze auf, um aktiv auf die Menschen zuzugehen, Kontakte zu knüpfen und Hilfe vor Ort anzubieten. Dafür werden zwei Teams bereitgestellt:

\begin{itemize}
	\item \textbf{Team \enquote{Konsum}}
	\item[] Als Hauptziel soll den Menschen dabei geholfen werden, Kontakte zu knüpfen und ihren Alltag aktiv zu gestalten. Das Team ist im Westen von Leipzig unterwegs.
	\item \textbf{Team \enquote{Wohnen}}
	\item[] Dieses Team konzentriert sich auf den Leipziger Norden und bietet Unterstützung und Hilfe bei unklaren oder heiklen Wohnsituationen.
\end{itemize}

Als Erweiterung kommt in Absprache mit den Teams auch der Hilfebus mit zum Einsatz. Im Zuge einer mobilen Grundversorgung, kann so den obdachlosen Menschen Essen, Getränke, Kleidung und Decken gegeben werden. Der Bus bietet zudem die Möglichkeit durch Kooperation mit ehrenamtlich arbeitenden Ärzten und Ärztinnen, als Anlaufstelle für Menschen mit gesundheitlichen Beschwerden zu fungieren. \citep{Safe}

\subsection{Weitere Angebote}

Dies ist nur ein kleiner Ausschnitt aus der Vielzahl an Hilfsangeboten. Es gibt dementsprechend noch weitere Anlaufstellen, die sich unterschiedlich spezialisiert haben.

Beispielsweise hat der Verein \enquote{TiMMi ToHelp} mehrere aktive Projekte, die sich allgemein mit den Themen Wohnungs- und Obdachlosigkeit, Armut und Wiedereingliederung befassen. Eine Tätigkeit ist die wöchentliche Verteilung von Care Bags, also von Lebensmitteln und Hygieneartikeln  an hilfebedürftige Menschen. In dieser Hinsicht können dem Verein verschiedene Sachspenden gegeben werden, die dann ihren Weg zu Menschen auf der Straße finden. \citep{TiMMi}

Zugehörig zum Krankenhaus St. Georg konzentriert sich der Verbund Gemeindenahe Psychiatrie auf die Behandlung von Menschen, die unter psychischen Erkrankungen bzw. psychosozialen Problemen leiden. Mit insgesamt fünf Standorten vereint das teilstationäre und ambulante Behandlungs- und Beratungszentrum die drei Betreuungsebenen Instanzambulanz, Tagesklinik und sozialpsychiatrischer Dienst und bietet zudem auch eine Peerberatung an, also Gespräche mit Menschen, die bereits Psychiatrieerfahrung haben. \citep{VGP}


\section{Bestehende Lösungsansätze}

Lösungsansätze für die Beseitigung von Wohnungslosigkeit bzw. zur Verbesserung der Situation für die Betroffenen kommen aus verschiedenen Richtungen. Dies liegt an den mehrschichtigen Problemen, die diese Menschengruppen umspannen. Während einige davon die Gründe von Wohnungslosigkeit im weitesten Sinne angehen, fokussieren sich andere auf konkrete Aspekte, die direkten Einfluss auf das Leben haben, wie z.B. das gesundheitliche Befinden.

Die Korrelation zwischen einer stabilen Unterkunft und der persönlichen Gesundheit steht im Kern des Ansatzes \enquote{Housing first}, der seinen Ursprung in den USA hat. Hierbei wird eine nicht vorhandene Behausung auf gleiche Ebene mit Krankheiten gesetzt, d.h. im Falle einer medizinischen Behandlung, sollte der Patient zunächst irgendwo unterkommen, um letztendlich die Erfolgschancen zu erhöhen. \citep{Srebnik.2013}

Das Assessment, Liaison and Early Referral Team (ALERT) des Krankenhauses St. Vincent's in Melbourne, Australien gehört zum klinikinternen Hospital Admission Risk Programm (HARP), welches eine allgemeine Reduzierung der Inanspruchnahme der Notfallaufnahme anstrebt. Das Team spezialisiert sich auf die Untergruppe der Patienten, die komplexe psychosoziale und medizinische Bedürfnisse aufweisen.

Das Team sah sich mit dem Problem konfrontiert, dass viele Termine für z.B. Nachuntersuchungen nicht eingehalten wurden, weil einige betroffene Personen keine festen Kontaktadressen bzw. -möglichkeiten wie E-Mail oder Telefon vorweisen konnten. Als einfache Lösung wurden günstige Mobiltelefone beschafft, indem auf bestimmte Brokerage-Fonds zugegriffen wurde. Diese wurden anschließend an die Patienten verteilt, was die Einhaltung von Terminen merkbar verbesserte. \citep{Davies.2018}

Wie bereits angemerkt, wird zudem der Einsatz des anonymen Behandlungsscheins weiter vorangetrieben. Der zentrale Kerngedanke dabei ist es, den Zugang zum hiesigen Gesundheitssystem für jeden zu ermöglichen, unabhängig von der jeweiligen Situation, in der sich die betroffenen Personen befinden. Dies geschieht, indem die Klärung des Versicherungsverhältnisses komplett unabhängig von einer Feststellung der Identität geschieht. \citep{Zanders.2022}

\todo{weiter ausbauen}