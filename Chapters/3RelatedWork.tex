%*****************************************
\chapter{Stand der Forschung}\label{ch:relatedWork}
%*****************************************

Medizinische Versorgung kann wie bereits erwähnt für viele Menschen ein heikles Thema sein. Insbesondere für Personen in Wohnungslosigkeit kann der Gang zum Arzt schon aufgrund von negativen Erfahrungen viel Überwindung kosten. Entsprechend müssen etwaige Hilfsangebote auf die Situation und die konkreten Personengruppen angepasst sein, damit ihr Angebot auch in Anspruch genommen wird.

\section{Anforderungen an die medizinische Versorgung von vulnerablen Gruppen}

Der Umgang mit vulnerablen Personengruppen ist schon länger ein Thema von bedeutendem Interesse und es wurden dahingehend auch Möglichkeiten untersucht, um den Zugang zu medizinischen Angeboten so einfach wie möglich zu gestalten. Im Zuge von Behandlungen ist zwischenmenschlicher Kontakt unausweichlich, wodurch das Personal erwartungsgemäß eine wichtige Schlüsselrolle beim Patientenkontakt einnimmt.

In \citet{Hwang.2014} wurden in Bezug auf die Arbeit mit Wohnungslosen einige Schlüsselpunkte zusammengestellt, die bei Gesundheitsdienstleistern mit beachtet werden sollten:

\begin{itemize}
	\item \textbf{Interpersonelle Beziehungen}
	\item[] Beim Umgang mit Wohnungslosen sollte immer entsprechender Respekt gezeigt werden, um die Würde des Gegenüber nicht zu verletzen. Man sollte den Menschen mit Wärme und Fürsorge entgegentreten und versuchen, gegenseitiges Vertrauen aufzubauen. In diesem Sinne könnten auch Peer-Beratungen, also Beratungen durch Menschen in (ehemals) ähnlichen Situationen, von Bedeutung sein.
	\item \textbf{Gemeinschaftsmittel}
	\item[] Dienstleister müssen mit den Hilfsprogrammen und Ressourcen vertraut sein, die den Patienten zur Verfügung stehen, oder in enger Zusammenarbeit mit Personal sein, die entsprechenden Vorkenntnisse besitzen. \newpage
	\item \textbf{Klinische Versorgung}
	\item[] Angepasste Richtlinien sollen den Dienstleistern helfen, den physischen und mentalen Gesundheitsbedürfnissen von Menschen in Wohnungslosigkeit gerecht zu werden. Dabei sollen Systeme etabliert werden, die Nachuntersuchungen sicherstellen und eine beiderseitige Kommunikation zwischen Krankenhäusern und gemeindenahen Anbietern erleichtern. Weiterhin sollten sogenannte Outreach-Programme mit hinzugezogen werden, um die Menschen aktiv zu erreichen.
	\item \textbf{Interessenvertretung}
	\item[] Dienstanbieter im Gesundheitswesen können sich für die Durchführung und Erhaltung von Maßnahmen einsetzen, die Verbesserungen der gesundheitlichen Situation von Wohnungslosen vorsehen.
\end{itemize}


\section{Hilfsangebote in Leipzig}

\todo{Aufgaben, Ziele, Organisation; Wie ausführlich?}

\subsection{Diakonie}

\subsection{CABL}

\subsection{Safe}

\subsection{Weitere Angebote}


\section{Bestehende Lösungsansätze}

\todo{Housing first, ALERT, ABS}