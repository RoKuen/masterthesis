%*****************************************
\chapter{Stand der Forschung}\label{ch:relatedWork}
%*****************************************

Medizinische Versorgung kann wie bereits erwähnt für viele Menschen ein heikles Thema sein. Insbesondere für Personen in Wohnungslosigkeit kann der Gang zum Arzt schon aufgrund von negativen Erfahrungen viel Überwindung kosten. Entsprechend müssen etwaige Hilfsangebote auf die Situation und die konkreten Personengruppen angepasst sein, damit ihr Angebot auch in Anspruch genommen wird. Die medizinische Dokumentation ist ebenfalls von diesen Umständen betroffen und muss entsprechend gestaltet werden.

\section{Dokumentation in der Medizin}

Medizinische Dokumentation erfüllt eine unabdingbare Rolle in der Gesundheitsversorgung und ist bei der Erfüllung einer Mehrzahl an Zielen beteiligt. In \citet{Leiner.2012} werden diese wie folgt aufgelistet:

\begin{itemize}
	\item Unterstützung der Patientenversorgung
	\item Erfüllen rechtlicher Erfordernisse
	\item Unterstützung der Administration
	\item Verbesserung der Informationsströme und des Betriebsablaufes
	\item Unterstützung des Qualitätsmanagements
	\item Unterstützung der klinisch-wissenschaftlichen Forschung
	\item Unterstützung der klinischen Aus- und Fortbildung
\end{itemize}

Diese Auswahl an Zielen verdeutlicht zunächst, dass medizinische Dokumentation nicht nur als Gedächtnisstütze dienen soll, sondern auch Einfluss auf zahlreiche andere Teilbereiche innerhalb einer Gesundheitseinrichtung nimmt. In Bezug auf die Versorgung von Wohnungslosen sind die Anforderungen an ein etwaiges Dokumentationssystem an diese Ziele angelehnt und sollten regelmäßig mit diesen abgeglichen werden.

Ähnlich dem Buch, geht es bei dieser Arbeit um den Bereich der klinischen Dokumentation. Dabei handelt es sich primär um die Handhabung von Informationen und Daten, die bei der medizinischen Versorgung konkreter Patienten und Patientinnen anfallen. Dazu gehören beispielsweise Anamnesebögen, Befunde, Diagnosen und Therapiepläne sowie generell Arztbriefe.

Es ist weiterhin festzuhalten, dass bei Überlegungen sowohl auf rechnergestützte als auch auf papierbasierte Methoden mit eingegangen werden sollte, da beide Vor- und Nachteile besitzen, die situationsbedingt von Relevanz sind. Der Einsatz von Rechnern kann z.B. eine weitestgehend ortsunabhängige Verfügbarkeit der Daten ermöglichen und Zeit bei der Verarbeitung der Daten einsparen, geht aber zumeist mit höheren Kosten und unter Umständen einer mühsameren Bedienung einher.


\subsection{Literaturrecherche}

Um entsprechende Publikationen und Statistiken zu erhalten, wurden verschiedene Suchmethoden verwendet. Dabei sind neben einfachen Websuchmaschinen (Google bzw. Google Scholar) auch Literaturdatenbanken wie z.B. PubMed zum Einsatz gekommen, um erste Publikationen zu finden, die als Quelle in Frage kommen. Während Einträge in etablierten Datenbanken bereits weitestgehend als seriös eingestuft werden können, sollten Ergebnisse aus allgemeineren, internetweiten Suchalgorithmen dahingehend näher untersucht werden.

Als Schlüsselwörter, die bei der Suche verwendet wurden, sind neben den folgenden Wörtern auch verschiedene Kombinationen miteineinander sowie den Umständen entsprechend auch deren englischen Übersetzungen zu Einsatz gekommen:

\begin{itemize}
	\item Wohnungslose, Obdachlose, Wohnungs- und Obdachlosigkeit (engl. \textit{homelessness})
	\item medizinische Dokumentation (engl. \textit{medical documentation})
	\item elektronische Patientenakte (engl. \textit{electronic medical record})
	\item Hilfsangebote, Wohnungslosenhilfe in Leipzig
	\item Umgang mit vulnerablen Gruppen
	\item Ansätze, Strategien, Interventionen
\end{itemize}

Gefundene Dokumente wurden zudem als Anhaltspunkte genommen, um mithilfe des Schneeballverfahrens weitere mögliche Quellen ausfindig zu machen. Dies ist vor allem hilfreich, da hierbei nicht nur verschiedene Studien zitiert werden, sondern auch auf Pressemitteilungen und Graue Literatur von verschiedenen relevanten Organisationen verwiesen wird.

Zuletzt konnten auch die Webseiten von Hilfsorganisationen, die auf einen der bereits genannten Wege gefunden wurden, auf weiterführende Links untersucht werden, denn viele nutzen diese Möglichkeit, um anzugehende Probleme oder zu erreichende Ziele besser zu beschreiben oder zu untermalen.

Eine Auflistung relevanter Publikationen, auf die in den einzelnen Kapiteln zurückgegriffen wird, ist in Tabelle \ref{tab:pub} zu finden.

\begin{sidewaystable}
\begin{table}[H]
	\centering
	\tymin=0.15\textheight
	\begin{tabulary}{\textheight}{LLLL}
		\toprule
		Titel& Inhalt& Kapitel&	Verweis\\
		\midrule
		The Medical Treatment of Homeless People&
		Gesundheitliche Situation von Wohnungslosen&
		\ref{sec:subject}&
		\citet{Kaduszkiewicz.2017}\\
		Psychische und somatische Gesundheit von wohnungslosen Menschen&
		Gesundheitliche Situation von Wohnungslosen&
		\ref{sec:subject}&
		\citet{DAE228829}\\
		Statistikbericht 2020: Zu Lebenslagen wohnungsloser und von Wohnungslosigkeit bedrohter Menschen in Deutschland -- Lebenslagenbericht&
		Aktuelle Statistiken zur Situation in Deutschland&
		\ref{sec:situation}&
		\citet{BAGW.2022}\\
		Health interventions for people who are homeless&
		Anforderungen an medizinische Versorgung von Wohnungslosen&
		\ref{sec:reqMedCare}&
		\citet{Hwang.2014}\\
		Wohnungslosigkeit in Deutschland aus europäischer Perspektive&
		Länderspezifische Situationen&
		\ref{sec:countryspec}&
		\citet{Busch-Geertsema.2018}\\
		Wohnungslosenpolitik in Europa. Nationale und europäische Strategien gegen Wohnungslosigkeit&
		Länderspezifische Situationen&
		\ref{sec:countryspec}&
		\citet{Busch-Geertsema.2012}\\
		Der anonyme Behandlungsschein -- von der Idee zur Umsetzung. Ein Handlungsleitfaden&
		Konzept zum \ac{ABS}&
		\ref{sec:approaches}&
		\citet{Zanders.2022}\\
		Homeless health care: meeting the challenges of providing primary care&
		Ansatz von ALERT&
		\ref{sec:approaches}&
		\citet{Davies.2018}\\
		Specialist medical centres for the homeless in Hamburg - diagnoses and reasons for treatment compared to general practice system (regular care system)&
		Schwerpunktpraxen in Hamburg&
		\ref{sec:approaches}&
		\citet{Leeden.2023}\\
		Safetynet Strategic Plan 2022 - 2024&
		Ansatz von Safetynet&
		\ref{sec:approaches}&
		\citet{Safetynet.2022}\\
		Electronic medical record implementation for a healthcare system caring for homeless people&
		Ansatz des BHCHP&
		\ref{sec:approaches}&
		\citet{Angoff.2019}\\
		\bottomrule
	\end{tabulary}
	\caption[Relevante Publikationen]{Eine Auflistung gefundener Publikationen von Relevanz}
	\label{tab:pub}
\end{table}
\end{sidewaystable}

\section{Bestehende Lösungsansätze}\label{sec:approaches}

Dokumentation ist ein essentieller Bestandteil des Aufgabenspektrums der verschiedenen Hilfsorganisationen, welche sich mit den Problemen und Anliegen von wohnungslosen Personen befassen. Im Folgenden werden einige Ansätze aufgelistet, die sich mit der medizinischen Versorgung dieser Personengruppe befassen sowie Methoden und Lösungen, die von einigen Organisationen in Hinblick auf die medizinische Dokumentation implementiert wurden.

\subsection{Housing First}

Die Korrelation zwischen einer stabilen Unterkunft und der persönlichen Gesundheit steht im Kern des Ansatzes \enquote{Housing first}, der seinen Ursprung in den USA hat. Hierbei wird eine nicht vorhandene Behausung auf gleiche Ebene mit Krankheiten gesetzt, d.h. im Falle einer medizinischen Behandlung, sollte der Patient zunächst irgendwo unterkommen, um letztendlich die Erfolgschancen zu erhöhen. 

Dieser eher allgemein formulierte Ansatz setzt noch keine konkreten Vorgaben oder Ziele bezüglich des eigentlichen Vorgehens oder der entstehenden bzw. benötigten Daten. Für weitere Systeme, die darauf aufbauen oder diesen Aspekt mit berücksichtigen, ist eine vorhergehende medizinische Dokumentation dennoch wichtig, um beispielsweise die Notwendigkeit einer Behausung adäquat einschätzen zu können. \citep{Srebnik.2013}

\subsection{Safetynet}

Der Wohlfahrtsverband \enquote{Safetynet} versucht Menschen, die am Rande der Gesellschaft stehen, eine hochwertige medizinische Versogung zu ermöglichen und fördert dahingehend auch ein Netzwerk von Gesundheitsdiensten, die mit wohnungs- und obdachlosen Personen arbeiten. Dafür werden mehrere Teams bereitgestellt, die medizinische Angebote in Dublin für diese Personengruppe anbieten. Diese werden zusammen mit den Zielen in einem entsprechenden Strategieplan vorgestellt. \citep{Safetynet.2022}

Ein \enquote{In-Reach Primary Care Team} versucht die gesundheitlichen Dienste näher an die Menschen zu bringen. Es arbeitet daher eng mit dem St. James oder der Notaufnahme des Mater Krankenhauses zusammen und kann dadurch hausärztliche sowie pflegerische Angebote fördern, die auch in den Notunterkünften oder bestimmten Drop-In-Zentren fortgeführt werden.

Für Obdachlose und generell Menschen, die keinen Zugang zu den Angeboten der Regelversorgung besitzen, ist außerdem ein mobiles Out-Reach-Team zuständig. Bestehend aus in Dublin ansässigen Allgemeinmedizinern und einigen Krankenschwestern, die zu Safetynet gehören, ist diese Einheit mit einem umgebauten Krankenwagen dreimal die Woche an Abenden unterwegs, um Leute auf der Straße aufzusuchen, zu beraten und zu betreuen.

Weitere Gruppierungen umfassen ein Team, welches den Housing-First-Ansatz verfolgt und halb-akute Behandlungen und Betreuungen mit unterstützt sowie mehrere Open-Access Kliniken, die Hilfe und Unterstützung für alle Menschen anbieten, die ansonsten keinen Zugang zu der primären Versorgung haben, wodurch auch ein großer Anteil der medizinischen Fachbereiche abgedeckt wird, um die unterschiedlichen gesundheitliche Anliegen behandeln zu können.

Zur Dokumentation wird auf ein webbasiertes Krankenakten- bzw. Dokumentationssystem zurückgegriffen, im Einsatz ist dafür das sogenannte \enquote{Socrates Praxisverwaltungssystem}. Das dahinterliegende Ziel ist es, die Gesundheitsversorgung qualitativ hochwertig zu gestalten, indem die Risiken für Mehrfachbehandlungen oder Fragmentierung der Dienste möglichst reduziert werden. Dies ist von besonderer Bedeutung, da Menschen, die in prekären Lagen leben, meist keine regulären Besuche bei z.B. Hausärzt:innen machen. Eine medizinische Vorgeschichte oder eine Auflistung verschriebener Medikamente müssen somit unabhängig vom jeweiligen behandelnden Arzt oder Ärztin vorliegen. \nocite{Safetynet}

\subsection{BHCHP}

Das \enquote{Boston Healthcare for the Homeless Program} (BHCHP) ist eines der größten freistehenden Gesundheitsversorgungsprogramme für Wohnungs- und Obdachlose und stellt für diese Bevölkerungsgruppe der US-amerikanischen Stadt Boston Präventions-, Behandlungs- und Betreuungsangebote bereit. Dazu werden an einer Großzahl an Standorten Kliniken betrieben, die sowohl medizinische als auch zahn-, verhaltens- und suchtmedizinische Leistungen anbieten.

Bereits 1996 wurde hier ein Patientenverwaltungssystem implementiert, welches auf die medizinische Versorgung der wohnungslosen Bevölkerung ausgerichtet war. Im Jahr 2015 entschied man sich dazu, zu einem neuen Anwendungssystem zu wechseln, namentlich \enquote{Epic}. Über einen Zwischenhändler konnte mit dem Hersteller ein kollaboratives Angebot zu gemeinsamen Nutzung ausgehandelt werden, sodass auch andere medizinische Hilfsorganisationen davon profitieren konnten.

Das Ziel dabei war es, sowohl die Funktionalität zu erhöhen und eine intraoperative Kommunikation innerhalb dieser flächendeckenden Implementierung zu ermöglichen. \citep{Angoff.2019}

\newpage
\subsection{ALERT}

Das Assessment, Liaison and Early Referral Team (ALERT) des Krankenhauses St. Vincent's in Melbourne, Australien gehört zum klinikinternen Hospital Admission Risk Programm (HARP), welches eine allgemeine Reduzierung der Inanspruchnahme der Notfallaufnahme anstrebt. ALERT spezialisiert sich dabei auf die Untergruppe der Patienten, die komplexe psychosoziale und medizinische Bedürfnisse aufweisen.

Das Team sah sich mit dem Problem konfrontiert, dass viele Termine für z.B. Nachuntersuchungen nicht eingehalten wurden, weil einige betroffene Personen keine festen Kontaktadressen bzw. -möglichkeiten wie E-Mail oder Telefon vorweisen konnten. Als relativ simple Lösung wurden günstige Mobiltelefone beschafft, indem auf bestimmte Brokerage-Fonds zugegriffen wurde. Diese wurden anschließend an die Patienten verteilt, was die Einhaltung von Terminen merkbar verbesserte. \citep{Davies.2018}

\subsection{Schwerpunktpraxen in Hamburg}

Ein konkreter Ansatz, der seit 2013 in Hamburg verfolgt wird, ist die medizinische Versorgung von Wohnungs- und Obdachlosen über sogenannte Schwerpunkt-Praxen. An bestimmten Tagen der Woche werden dort hausärztliche Sprechstunden angeboten, die speziell für vulnerable Gruppen bzw. Personen ohne Versicherungsschutz ausgerichtet sind. An zwei der insgesamt drei Standorten können Betroffene zudem auch eine psychiatrische Beratung in Anspruch nehmen.

Die dort tätigen Hausärzte und Psychiater sind in Hamburg niedergelassen und arbeiten in den dortigen Praxen. Da sie in diesem Zusammenhang nur die Grundversorgung übernehmen, wird für Fälle, die weiteren Maßnahmen bedürfen, ein Kontakt mit den Institutionen der Regelversorgung hergestellt.

Zu diesem Ansatz gibt es jedoch keine detaillierten Informationen bezüglich Umfang oder Inhalt der Dokumentation. Unter der Annahme, dass die Schwerpunktpraxen in ähnlicher Weise wie eine Hausarztpraxis vorgehen, könnte man davon ausgehen, dass Informationen zu den vorstellig gewordenen Personen fallbasiert in einem Praxisverwaltungssystem gespeichert werden. \citep{Leeden.2023}

\subsection{Anonymer Behandlungsschein}

Wie bereits angemerkt, wird zudem der Einsatz des anonymen Behandlungsscheins (\acs{ABS}) weiter vorangetrieben. Der zentrale Kerngedanke dabei ist es, den Zugang zum hiesigen Gesundheitssystem für jeden zu ermöglichen, unabhängig von den jeweiligen Situationen, in denen sich die betroffenen Personen befinden. 

Es handelt sich hierbei jedoch meist nur um einen kurzfristigen Zugang zum Gesundheitssystem in Deutschland. Die ausstellenden Clearingstellen, z.B. \ac{CABL} oder der \ac{AKST}, versuchen dies für die Betroffenen auch langfristig möglich zu machen, weshalb beim Clearing versucht wird, einen anderen Kostenträger ausfindig zu machen oder die Personen über eine Krankenkasse abzusichern. Der \acs{ABS} ist damit nur ein Ansatzpunkt bei dem Versuch, medizinische Versorgung niedrigschwellig zur Verfügung zu stellen.

Zur Evaluation dieses Projektes werden dabei auch Daten zu Fallzahlen oder Kosten erhoben, wobei auch eine Dokumentation typischer sowie problematischer Fälle in Betracht gezogen wird. \citep{Zanders.2022}