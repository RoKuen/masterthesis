%*****************************************
\chapter{Stand der Forschung}\label{ch:relatedWork}
%*****************************************

Medizinische Versorgung kann wie bereits erwähnt für viele Menschen ein heikles Thema sein. Insbesondere für Personen in Wohnungslosigkeit kann der Gang zum Arzt schon aufgrund von negativen Erfahrungen viel Überwindung kosten. Entsprechend müssen etwaige Hilfsangebote auf die Situation und die konkreten Personengruppen angepasst sein, damit ihr Angebot auch in Anspruch genommen wird. Eine etwaige Dokumentation ist ebenfalls von diesen Umständen betroffen und muss entsprechend gestaltet werden.

\section{Dokumentation in der Medizin}

Die Aufgabe der medizinischen Dokumentation ist wie bereits erwähnt von großer Bedeutung, weshalb es in dieser Hinsicht reichlich Werke gibt, die sich diesem Feld widmen. Die Erklärungen reichen dabei von den allgemeinen Grundbegriffen bis hin zu einzelnen Ordnungssystemen, die in der Medizin Verwendung finden, wie z.B. die internationale Klassifikation der Krankheiten (ICD) oder die Operationen- und Prozedurenschlüssel (OPS).

Mit dem Fokus auf den Bereich der klinischen Dokumentation wird in dem Lehrbuch \citet{Leiner.2012} bereits im ersten Kapitel eine Zusammenstellung von inhaltlichen Zielen der medizinischen Dokumentation gegeben, auf die im Laufe des Buches regelmäßig zurückgegriffen wird. Dazu gehören:

\begin{itemize}
	\item Unterstützung der Patientenversorgung
	\item Erfüllen rechtlicher Erfordernisse
	\item Unterstützung der Administration
	\item Verbesserung der Informationsströme und des Betriebsablaufes
	\item Unterstützung des Qualitätsmanagements
	\item Unterstützung der klinisch-wissenschaftlichen Forschung
	\item Unterstützung der klinischen Aus- und Fortbildung
\end{itemize}

Diese Auswahl an Zielen verdeutlicht zunächst, dass medizinische Dokumentation nicht nur als Gedächtnisstütze dienen soll, sondern auch Einfluss auf zahlreiche andere Teilbereiche innerhalb einer Gesundheitseinrichtung nimmt. In Bezug auf die Versorgung von Wohnungslosen sind die Anforderungen an ein etwaiges Dokumentationssystem an diese Ziele angelehnt und sollten regelmäßig mit diesen abgeglichen werden.

Ähnlich dem Buch, geht es bei dieser Arbeit um den Bereich der klinischen Dokumentation. Dabei handelt es sich primär um die Handhabung von Informationen und Daten, die bei der medizinischen Versorgung konkreter Patienten und Patientinnen anfallen. Dazu gehören beispielsweise Anamnesebögen, Befunde, Diagnosen und Therapiepläne sowie generell Arztbriefe.

Es ist weiterhin festzuhalten, dass bei Überlegungen sowohl auf rechnergestützte als auch auf papierbasierte Methoden mit eingegangen werden sollte, da beide Vor- und Nachteile besitzen, die situationsbedingt von Relevanz sind. Der Einsatz von Rechnern kann z.B. eine weitestgehend ortsunabhängige Verfügbarkeit der Daten ermöglichen und Zeit bei der Verarbeitung der Daten einsparen, geht aber zumeist mit höheren Kosten und unter Umständen einer mühsameren Bedienung einher.


\section{Bestehende Lösungsansätze}

Lösungsansätze für die Beseitigung von Wohnungslosigkeit bzw. zur Verbesserung der Situation für die Betroffenen kommen aus verschiedenen Richtungen. Dies liegt an den mehrschichtigen Problemen, die diese Menschengruppen umspannen. Während einige davon die Gründe von Wohnungslosigkeit im weitesten Sinne angehen, fokussieren sich andere auf konkrete Aspekte, die direkten Einfluss auf das Leben haben, wie z.B. das gesundheitliche Befinden.

\subsection{Housing First}

Die Korrelation zwischen einer stabilen Unterkunft und der persönlichen Gesundheit steht im Kern des Ansatzes \enquote{Housing first}, der seinen Ursprung in den USA hat. Hierbei wird eine nicht vorhandene Behausung auf gleiche Ebene mit Krankheiten gesetzt, d.h. im Falle einer medizinischen Behandlung, sollte der Patient zunächst irgendwo unterkommen, um letztendlich die Erfolgschancen zu erhöhen. 

Dieser eher allgemein formulierte Ansatz setzt noch keine konkreten Vorgaben oder Ziele bezüglich des eigentlichen Vorgehens oder der entstehenden bzw. benötigten Daten. Für weitere Systeme, die darauf aufbauen oder diesen Aspekt mit berücksichtigen, ist eine vorhergehende medizinische Dokumentation dennoch wichtig, um beispielsweise die Notwendigkeit einer Behausung adäquat einschätzen zu können. \citep{Srebnik.2013}

\subsection{Safetynet}

Der Wohlfahrtsverband \enquote{Safetynet} versucht Menschen, die am Rande der Gesellschaft stehen, eine hochwertige medizinische Versogung zu ermöglichen und fördert dahingehend auch ein Netzwerk von Gesundheitsdiensten, die mit wohnungs- und obdachlosen Personen arbeiten. Dafür werden mehrere Teams bereitgestellt, die medizinische Angebote in Dublin für diese Personengruppe anbieten. 

Ein \enquote{In-Reach Primary Care Team} versucht die gesundheitlichen Dienste näher an die Menschen zu bringen. Es arbeitet daher eng mit dem St. James oder der Notaufnahme des Mater Krankenhauses zusammen und kann dadurch hausärztliche sowie pflegerische Angebote fördern, die auch in den Notunterkünften oder bestimmten Drop-In-Zentren fortgeführt werden.

Für Obdachlose und generell Menschen, die keinen Zugang zu den Angeboten der Regelversorgung besitzen, ist außerdem ein mobiles Out-Reach-Team zuständig. Bestehend aus in Dublin ansässigen Allgemeinmedizinern und einigen Krankenschwestern, die zu Safetynet gehören, ist diese Einheit mit einem umgebauten Krankenwagen dreimal die Woche an Abenden unterwegs, um Leute auf der Straße aufzusuchen, zu beraten und zu betreuen.

Weitere Gruppierungen umfassen ein Team, welches den Housing-First-Ansatz verfolgt und halb-akute Behandlungen und Betreuungen mit unterstützt sowie mehrere Open-Access Kliniken, die Hilfe und Unterstützung für alle Menschen anbieten, die ansonsten keinen Zugang zu der primären Versorgung haben, wodurch auch ein großer Anteil der medizinischen Fachbereiche abgedeckt wird, um die unterschiedlichsten gesundheitliche Anliegen behandeln zu können.

Zur Dokumentation wird auf ein webbasiertes Krankenaktensystem zurückgegriffen, im Einsatz ist dafür das sogenannte \enquote{Socrates Praxisverwaltungssystem}. Das dahinterliegende Ziel ist es, die Gesundheitsversorgung qualitativ hochwertig zu gestalten, indem die Risiken für Mehrfachbehandlungen oder Fragmentierung der Dienste möglichst reduziert werden. Dies ist von besonderer Bedeutung, da Menschen, die in prekären Lagen leben, meist keine regulären Besuche bei z.B. Hausärzt:innen machen. Eine medizinische Vorgeschichte oder eine Auflistung verschriebener Medikamente müssen somit unabhängig vom jeweiligen behandelnden Arzt oder Ärztin vorliegen. \citep{Safetynet.2022} \nocite{Safetynet}

\subsection{ALERT}

Das Assessment, Liaison and Early Referral Team (ALERT) des Krankenhauses St. Vincent's in Melbourne, Australien gehört zum klinikinternen Hospital Admission Risk Programm (HARP), welches eine allgemeine Reduzierung der Inanspruchnahme der Notfallaufnahme anstrebt. ALERT spezialisiert sich dabei auf die Untergruppe der Patienten, die komplexe psychosoziale und medizinische Bedürfnisse aufweisen.

Das Team sah sich mit dem Problem konfrontiert, dass viele Termine für z.B. Nachuntersuchungen nicht eingehalten wurden, weil einige betroffene Personen keine festen Kontaktadressen bzw. -möglichkeiten wie E-Mail oder Telefon vorweisen konnten. Als relativ simple Lösung wurden günstige Mobiltelefone beschafft, indem auf bestimmte Brokerage-Fonds zugegriffen wurde. Diese wurden anschließend an die Patienten verteilt, was die Einhaltung von Terminen merkbar verbesserte. \citep{Davies.2018}

\subsection{Schwerpunktpraxen in Hamburg}

Ein konkreter Ansatz, der seit 2013 in Hamburg verfolgt wird, ist die medizinische Versorgung von Wohnungs- und Obdachlosen über sogenannte Schwerpunkt-Praxen. An bestimmten Tagen der Woche werden dort hausärztliche Sprechstunden angeboten, die speziell für vulnerable Gruppen bzw. Personen ohne Versicherungsschutz ausgerichtet sind. An zwei der insgesamt drei Standorten können Betroffene zudem auch ein psychiatrische Beratung in Anspruch nehmen.

Die dort tätigen Hausärzte und Psychiater sind in Hamburg niedergelassen und arbeiten in den dortigen Praxen. Da sie in diesem Zusammenhang nur die Grundversorgung übernehmen, wird für Fälle, die weiteren Maßnahmen bedürfen, ein Kontakt mit den Institutionen der Regelversorgung hergestellt. \citep{Leeden.2023}

\subsection{Anonymer Behandlungsschein}

Wie bereits angemerkt, wird zudem der Einsatz des anonymen Behandlungsscheins (\acs{ABS}) weiter vorangetrieben. Der zentrale Kerngedanke dabei ist es, den Zugang zum hiesigen Gesundheitssystem für jeden zu ermöglichen, unabhängig von der jeweiligen Situation, in der sich die betroffenen Personen befinden. 

Es handelt sich hierbei jedoch meist nur um einen kurzfristigen Zugang zum Gesundheitssystem in Deutschland. Die ausstellenden Clearingstellen, z.B. \ac{CABL} oder der \ac{AKST}, versuchen dies für die Betroffenen auch langfristig möglich zu machen, weshalb beim Clearing versucht wird, einen anderen Kostenträger ausfindig zu machen oder die Personen über eine Krankenkasse abzusichern. Der \acs{ABS} ist damit nur ein Ansatzpunkt bei dem Versuch, medizinische Versorgung niedrigschwellig zur Verfügung zu stellen.

Zur Evaluation dieses Projektes werden dabei auch Daten zu Fallzahlen oder Kosten erhoben, wobei auch eine Dokumentation typischer sowie problematischer Fälle in Betracht gezogen wird. \citep{Zanders.2022}
\todo{anpassen um leere Seiten zu vermeiden}