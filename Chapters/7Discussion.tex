%*****************************************
\chapter{Diskussion}\label{ch:discussion}
%*****************************************

Nach erfolgter Analyse des Leipziger Informationssystems in Bezug auf die medizinische Versorgung von Wohnungslosen und anderen Personen aus vulnerablen Gruppen, sowie der damit verbundenen Dokumentation, erschließt sich ein Gesamtbild von einer Mehrzahl verschiedener Hilfsangebote, die unterschiedliche Tools und Anwendungen im Einsatz haben, welche je nach vorliegender Situation gewählt wurden. Diese Angebote sind auf die spezifischen Bedürfnisse der Zielgruppe angepasst, indem den Hilfesuchenden aufsuchend oder an mehreren Standorten verteilt Kontakte zur Verfügung gestellt werden, die ihnen bei vielfältigen Problemen zu Seite stehen.

Allerdings bauen einige dieser Angebote, vor allem in Bezug auf die Behandlung somatischer Beschwerden, auf die Mithilfe von ehrenamtlichen Ärzt:innen auf. Diese Abhängigkeit und die daraus resultierende Notwendigkeit, diese zusätzliche Arbeitslast möglichst zu reduzieren, damit die Personen weiterhin zur Unterstützung bereit sind, führt zu der Annahme, dass die Aufgabe der Dokumentation in diesem Umfeld eher eine untergeordnete Rolle spielt. Daten, die in diesem Zusammenhang erfasst werden, dienen in einigen Fällen primär dem Zweck der statistischen Auswertung, sodass beispielsweise konkrete Angaben zu Fallzahlen gemacht werden können.

Bei der Implementierung eines neuen Dokumentationssystems sind die Punkte Verfügbarkeit und Benutzerfreundlichkeit besonders wichtig. Es sollte nicht nur an den unterschiedlichsten Orten abrufbar sein, sondern auch ein Gleichgewicht zwischen Umfang und Aufwand der Dokumentation finden. Webbasierte Ansätze, wie sie bereits beschrieben wurden, sollten sich für diesen ersten Punkt gut eignen, da hierbei unterschiedliche (Mobil-)Geräte zum Zugriff verwendet werden können. Medizinisch relevante Dokumente können dann über einen Cloud-Service abgespeichert werden, damit sie an den richtigen Orten und zu den richtigen Zeiten einsehbar sind.