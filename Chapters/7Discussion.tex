%*****************************************
\chapter{Diskussion}\label{ch:discussion}
%*****************************************

Im Rahmen dieser Arbeit wurden Informationen zum Leipziger Informationssystem gesammelt, welches sich mit der medizinischen und sozialarbeitsbezogenen Dokumentation bei Wohnungslosen befasst. Die Entscheidung, hierfür halbstrukturierte Interviews als Methode zu verwenden, scheint für diesen Zweck sehr gut geeignet gewesen zu sein. Im Gespräch mit Personen, welche die unterschiedlichen Aufgaben im Umfeld der Versorgung wohnungs- und obdachloser Menschen erfüllen, konnten Erkenntnisse über die Verfahrensweisen und den Fluss an Informationen gewonnen werden. Aufgetretene Probleme waren minimal und meist organisatorische Natur, wodurch sie relativ schnell gelöst werden konnten.

Die Aufnahme und anschließende Transkription des Gesprochenen war für das Anfertigen der Modelle äußerst hilfreich. Obwohl durch die Verwendung des Sprachmodells \enquote{Whisper} der Arbeitsaufwand effizient eingeteilt werden konnte, war die Erstellung der Transkripte mit einem nicht unerheblichen Zeitaufwand verbunden. Die verwendeten Modelle eigneten sich gut, um die Eigenschaften des Leipziger Informationssystems sowie dessen Stärken und Schwächen zu beleuchten, wobei man den zeitlichen Aspekt der Kommunikationsdiagramme unter Umständen als redundant ansehen könnte.

Das erste Ziel, welches mit dieser Arbeit angestrebt wurde, namentlich die Systemanalyse und -bewertung des oben genannten Systems, kann damit als vollständig erreicht angesehen werden. Die erzielten Ergebnisse sollten sich für etwaige Projekte, die sich in diesem Umfeld bewegen, als hilfreich erweisen.

Bezüglich des zweiten Ziels, der Gegenüberstellung mit bestehenden Ansätzen, kam es zu dem Umstand, dass Informationen bezüglich der Dokumentation meist nicht sehr detailliert waren, wodurch ein Vergleich nur mit einigen ausgewählten Ansätzen sinnvoll erschien. Ebenso musste die Anzahl der Vergleichskriterien eingeschränkt werden, da viele Punkte ein tieferes Verständnis der Verfahren voraussetzen, als dies hier der Fall war. Im Ergebnis konnten dennoch aussagekräftige Vergleiche gezogen werden.

Die Analyse des Leipziger Informationssystems zur medizinischen Versorgung von Wohnungslosen und anderen Menschen in prekären Lebenslagen und der damit verbundenen Dokumentation ergibt ein Gesamtbild einer Vielzahl unterschiedlicher Hilfsangebote, die unterschiedliche Tools und Anwendungen im Einsatz haben, welche je nach vorliegender Situation gewählt wurden. Diese Angebote sind auf die spezifischen Bedürfnisse der Zielgruppe angepasst, indem den Hilfesuchenden aufsuchend oder an mehreren Standorten verteilt Kontakte zur Verfügung gestellt werden, die ihnen bei vielfältigen Problemen zu Seite stehen.

Allerdings bauen einige dieser Angebote, vor allem in Bezug auf die Behandlung somatischer Beschwerden, auf die Mithilfe von ehrenamtlichen Ärzt:innen auf. Diese Abhängigkeit und die daraus resultierende Notwendigkeit, diese zusätzliche Arbeitslast möglichst zu reduzieren, damit die Personen weiterhin zur Unterstützung bereit sind, führt zu der Annahme, dass die Aufgabe der Dokumentation in diesem Umfeld eher eine untergeordnete Rolle spielt. Daten, die in diesem Zusammenhang erfasst werden, dienen in einigen Fällen primär dem Zweck der statistischen Auswertung, sodass beispielsweise konkrete Angaben zu Fallzahlen gemacht werden können.

Bei der Implementierung eines neuen Dokumentationssystems sind die Punkte Verfügbarkeit und Benutzerfreundlichkeit besonders wichtig. Es sollte nicht nur von verschiedenen Orten aus zugänglich sein, sondern auch ein Gleichgewicht zwischen Umfang und Aufwand der Dokumentation finden. Webbasierte Ansätze, wie sie bereits beschrieben wurden, sollten sich für diesen ersten Punkt gut eignen, da hierbei unterschiedliche (mobile) Geräte zum Zugriff verwendet werden können. Medizinisch relevante Dokumente können dann über einen Cloud-Service abgespeichert werden, damit sie an den richtigen Orten und zu den richtigen Zeiten abrufbar sind.

Einige befragte Personen haben während den Interviews erwähnt, dass die medizinische Grundversorgung Obdachloser in den staatlichen Aufgabenbereich fallen bzw. weniger von ehrenamtlicher Tätigkeit abhängig sein sollte. Deshalb sollten zukünftige Entwicklungen in dieser Richtung bei der Konzeption eines Dokumentationssystems in diesem Bereich berücksichtigt werden.

Die medizinische Versorgung wohnungsloser Menschen stellt schließlich besondere Anforderungen, auch an die Dokumentation. Ein entsprechendes Dokumentationssystem muss daher an die Bedürfnisse der Leistungserbringer und der Betroffenen angepasst werden. Auf diese Weise kann letztlich eine Verbesserung der gesundheitlichen Situation dieser Personengruppe erreicht werden.