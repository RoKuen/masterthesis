%*****************************************
\chapter{Diskussion}\label{ch:discussion}
%*****************************************

Im Rahmen dieser Arbeit wurden Informationen zum Leipziger Informationssystem gesammelt, welches sich mit der medizinischen und sozialarbeitsbezogenen Dokumentation bei Wohnungslosen befasst. Die Entscheidung, hierfür halbstrukturierte Interviews als Methode zu verwenden, scheint für diesen Zweck sehr gut geeignet gewesen zu sein. Im Gespräch mit Personen, welche die unterschiedlichen Aufgaben im Umfeld der Versorgung wohnungs- und obdachloser Menschen erfüllen, konnten Erkenntnisse über die Verfahrensweisen und den Fluss an Informationen gewonnen werden. Aufgetretene Probleme waren minimal und meist organisatorische Natur, wodurch sie relativ schnell gelöst werden konnten.

Die Aufnahme und anschließende Transkription des Gesprochenen war für das Anfertigen der Modelle äußerst hilfreich. Zwar konnte durch die Verwendung des Sprachmodells \enquote{Whisper} der Arbeitsaufwand effizient eingeteilt werden, dennoch wurde für das Erstellen der Transkripte eine nicht unwesentliche Menge an Zeit benötigt. Die verwendeten Modelle eigneten sich gut, um die Eigenschaften des Leipziger Informationssystems sowie dessen Stärken und Schwächen zu beleuchten, wobei man den zeitlichen Aspekt der Kommunikationsdiagramme unter Umständen als redundant ansehen könnte.

Das erste Ziel, welches mit dieser Arbeit angestrebt wurde, namentlich die Systemanalyse und -bewertung des oben genannten Systems, kann damit als vollständig erreicht angesehen werden. Die erzielten Ergebnisse sollten sich für etwaige Projekte, die sich in diesem Umfeld bewegen, als hilfreich erweisen.

Bezüglich des zweiten Ziels, der Gegenüberstellung mit bestehenden Ansätzen, kam es zu dem Umstand, dass Informationen bezüglich der Dokumentation meist nicht sehr detailliert waren, wodurch ein Vergleich nur mit einigen ausgewählten Ansätzen sinnvoll erschien. Die Anzahl der Vergleichskriterien musste auf ähnliche Weise begrenzt werden, da viele Punkte ein tieferes Verständnis über die Verfahrensweisen erfordern, als hierbei angedacht war. Schlussendlich konnte dennoch ein passender Vergleich gezogen werden.

Nach erfolgter Analyse des Leipziger Informationssystems in Bezug auf die medizinische Versorgung von Wohnungslosen und anderen Personen aus vulnerablen Gruppen, sowie der damit verbundenen Dokumentation, erschließt sich ein Gesamtbild von einer Mehrzahl verschiedener Hilfsangebote, die unterschiedliche Tools und Anwendungen im Einsatz haben, welche je nach vorliegender Situation gewählt wurden. Diese Angebote sind auf die spezifischen Bedürfnisse der Zielgruppe angepasst, indem den Hilfesuchenden aufsuchend oder an mehreren Standorten verteilt Kontakte zur Verfügung gestellt werden, die ihnen bei vielfältigen Problemen zu Seite stehen.

Allerdings bauen einige dieser Angebote, vor allem in Bezug auf die Behandlung somatischer Beschwerden, auf die Mithilfe von ehrenamtlichen Ärzt:innen auf. Diese Abhängigkeit und die daraus resultierende Notwendigkeit, diese zusätzliche Arbeitslast möglichst zu reduzieren, damit die Personen weiterhin zur Unterstützung bereit sind, führt zu der Annahme, dass die Aufgabe der Dokumentation in diesem Umfeld eher eine untergeordnete Rolle spielt. Daten, die in diesem Zusammenhang erfasst werden, dienen in einigen Fällen primär dem Zweck der statistischen Auswertung, sodass beispielsweise konkrete Angaben zu Fallzahlen gemacht werden können.

Bei der Implementierung eines neuen Dokumentationssystems sind die Punkte Verfügbarkeit und Benutzerfreundlichkeit besonders wichtig. Es sollte nicht nur an den unterschiedlichsten Orten abrufbar sein, sondern auch ein Gleichgewicht zwischen Umfang und Aufwand der Dokumentation finden. Webbasierte Ansätze, wie sie bereits beschrieben wurden, sollten sich für diesen ersten Punkt gut eignen, da hierbei unterschiedliche (Mobil-)Geräte zum Zugriff verwendet werden können. Medizinisch relevante Dokumente können dann über einen Cloud-Service abgespeichert werden, damit sie an den richtigen Orten und zu den richtigen Zeiten einsehbar sind.

Bei einigen Interviews hab die befragten Personen angegeben, dass die medizinische Grundversorgung Obdachloser eine staatliche Aufgabe werden sollte. Dahingehend sollte bei der Konzeption eines Dokumentationssystems in diesem Umfeld auch zukünftige Veränderungen in dieser Richtung mit bedacht werden.

Schlussendlich kann jedoch gesagt werden, dass für die medizinische Versorgung von Wohnungslosen spezielle Anforderungen, unter anderem auch an die Dokumentation, gestellt werden. Ein entsprechendes Dokumentationssystem muss demnach an die Bedürfnisse der Leistungserbringer und der Betroffenen angepasst sein. Auf diese Weise kann anschließend eine Verbesserung der gesundheitlichen Situation dieser Personengruppe in Aussicht gestellt werden.