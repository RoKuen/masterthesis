%************************************************
\chapter{Einleitung}\label{ch:introduction}
%************************************************

Die Aufgabe der Dokumentation ist in der heutigen Welt in vielen Bereichen ein essentieller Baustein im Ablauf von Prozessen. Insbesondere in der medizinischen Versorgung wird diesem Aspekt eine große Bedeutung zugeordnet. Sie dient nicht nur als Gedächtnisstütze für medizinische Fachkräfte sondern ist auch relevant bei der Weiterbehandlung von Patient:innen sowie in anderen Umfeldern wie Forschung oder Krankenversicherungen. Sie hat zudem eine gewisse Bedeutung, wenn es um den Nachweis von Behandlungsfehlern geht. Etwaige patientenbezogene Informationen müssen dahingehend sorgfältig erfasst, gespeichert und gepflegt werden. Die Anforderungen an ein zugehöriges medizinisches Dokumentationssystem sind dementsprechend hoch. \citep{DAE45930}

\section{Gegenstand}\label{sec:subject}

Der Zugang zum deutschen Gesundheitssystem ist für viele Bürger und Bürgerinnen allgemein klar: Man geht mit seinen Beschwerden bzw. seinem Anliegen in eine Arztpraxis, weist sich mit seiner Gesundheitskarte aus und wird anschließend beraten und behandelt. In kritischen Fällen geht man direkt zur zentralen Notaufnahme eines Krankenhauses oder ruft einen Krankenwagen. Unabhängig von der konkreten Situation ist ein Nachweis der Identität oder Krankenversicherung ein integraler Bestandteil dieses Prozesses, der den Verweis auf eine ggf. bestehende Dokumentation über den betroffenen Patienten erbringt.

Für viele Menschen stellt dies aber meist ein Problem dar und sie meiden den Gang zur Arztpraxis. Die Grunde hierfür sind vielfältig und reichen von einer fehlenden bzw. nicht nachweisbaren Mitgliedschaft bei einer Krankenkasse bis zu persönlichen und psychologischen Gründen, wie Scham oder Angst vor Diskriminierung, was mache veranlasst, ihren Namen nicht preisgeben zu wollen. \citep{Kaduszkiewicz.2017} Manche Krankheit oder Komplikation, die im Normalfall schnell erkannt und behandelt werden kann, wird ignoriert und hat das Risiko, ein Ernstfall zu werden. So ist eine Notaufnahme meist die erste Anlaufstelle für die Betroffenen, was ärztliches Personal und andere Mitarbeitende weiter belastet.

Obwohl die Grundprinzipien des deutschen Gesundheitssystems darauf abzielen, jedem Menschen eine Behandlung zu ermöglichen, gibt es eine nicht vernachlässigbare Menge an Menschen, denen der Zugang schwer fällt oder nicht möglich ist. Der Großteil dieser Gruppe besteht aus Wohnungslosen, was zum einen Menschen einschließt, die keine eigene Wohnung haben, zum anderen sind hierbei auch Personen gemeint, die keinen Aufenthaltsstatus haben und sich per se illegal in Deutschland aufhalten. Nach Angaben der \citet{BAGWohnungslosenhilfe.2021} liegt die Jahresgesamtzahl aller wohnungslosen Menschen in Deutschland in 2020 bei geschätzt 417.000, wovon ca. 45.000 im Laufe eines Jahres ohne jegliche Unterkunft und somit auf der Straße leben.

Hinzu kommt, dass bei Personen in diesen Umfeldern sowohl somatische als auch psychische Erkrankungen stark prävalent sind. So werden kardiovaskuläre Erkrankungen sowie Krankheiten der Leber und Lunge besonders häufig berichtet und etwa zwei Drittel der Menschen zeigen Hinweise für eine mögliche unbekannte psychische Störung. \citep{DAE228829}

Um die medizinische Versorgung und den allgemeinen Kontakt zu Institutionen des Gesundheitswesens zu erleichtern, wurden in vielen Städten Deutschlands Hilfsangebote für Wohnungslose eingerichtet. In Leipzig organisiert der \ac{CABL} Möglichkeiten für Sprechstunden mit ehrenamtlich arbeitenden Ärzten und Ärztinnen. Hierzu werden nicht nur bereits bestehende Anlaufstellen wie z.B. die Leipziger OASE herangezogen, sondern auch mobile Optionen wie ein Behandlungsbus oder die privaten PKWs der Ärzt:innen bereitgestellt.

Die medizinische Dokumentation erfolgt in solchen Fällen jedoch meist nur rudimentär, wobei auch immer wieder Probleme mit der Identifikation der zu behandelnden Personen auftreten, da entsprechende Ausweisdokumente fehlen oder die Aussage aufgrund persönlicher Gründe verweigert wird. Um diesen Wunsch der Anonymität weitestgehend gerecht zu werden, soll mithilfe des anonymen Behandlungsscheins (\acs{ABS}) jedem Patienten und jeder Patientin ein Pseudonym zugeteilt werden. Dieser Schein soll anschließend bei anderen Institutionen eine gleichwertige Ausweisfunktion wie z.B. ein Versicherungsnachweis tragen. \citep{CABL}


\section{Problemstellung}

Eine mangelhafte Dokumentation kann vor allem im medizinischen Bereich gravierende Folgen haben, die von Zeitproblemen bei den Prozessen bis hin zu gesundheitlichen Konflikten, z.B. im Zusammenhang mit Medikamenten, bei den beteiligten Patienten reichen. \citep{Silvestre.2017} Sollten wichtige Informationen fehlen, so müssen diese mithilfe von Doppeluntersuchungen erneut beschafft werden, was sowohl Kostenträger, Leistungserbringer und die betroffenen Patient:innen weiter belastet.

Der Grund für diese Notwendigkeit besteht darin, dass eine patientenspezifische Krankheitshistorie und Auflistung durchgeführter Prozeduren für die Diagnose und Therapieauswahl unerlässlich sind. Eine ungenaue bzw. fehlerhafte Aufzeichnung wichtiger Informationen kann dabei u.a. durch den Gebrauch suboptimaler Dokumentationssysteme geschehen. 

Ein weiterer Punkt ist, dass bestehende Versorgungslücken im Gesundheitswesen, wie z.B. bei der Versorgung Wohnungsloser, zwar von verschiedenen gemeinnützigen Organisationen angegangen werden, diese jedoch meist nicht den vollen Versicherungsschutz ersetzen können, da sie neben Spendengeldern auch von den zeitlichen Kapazitäten ehrenamtlicher Mitarbeiter abhängig sind. \citep[S. 6]{Zanders.2022}

Zur Entwicklung besserer und spezialisierter Dokumentationssysteme wird eine grundlegende Bibliothek an Wissen und Erfahrungsberichten benötigt. Diese ist jedoch genau in dem beschriebenen Fall meist unvollständig und lückenhaft, was das Erkennen von Problemursachen und den generellen Überblick über das Themengebiet erschwert. Weiterhin muss bedacht werden, dass es eventuell bereits existierende Lösungen hierzu gibt, die zur Betrachtung mit hinzugezogen oder mittels kleinerer Erweiterungen an die Situation vor Ort angepasst werden können. 

Eine solche Zusammenstellung relevanter Informationen muss vor Entwicklungsbeginn bereitliegen. Bezüglich dessen sollen in dieser Arbeit die folgenden Probleme behandelt werden:

\begin{itemize}
\item \textbf{P1: Unvollständige Informationen über die medizinische und sozialarbeitsbezogene Dokumentation bei Wohnungslosen}
	\begin{itemize}
		\item[] Um Dokumentationssysteme entwickeln zu können, die den Bedürfnissen in diesem Umfeld gerecht werden, wird eine detaillierte Übersicht zu den Inhalten, Akteuren, Werkzeugen und Prozessen der Dokumentation benötigt, sowohl im medizinischen Sinne als auch im Kontext der Sozialberatung mit Personen von Hilfsorganisationen, in der die Anbindung an die Regelversorgung unterstützt wird. Nach aktuellem Stand sind die vorliegenden Informationen sehr lückenhaft.
	\end{itemize}
\item \textbf{P2: Unklarheit bezüglich etablierten Lösungen}
	\begin{itemize}
		\item[] Es ist noch nicht klar, ob es außerhalb Leipzig bzw. international bereits etablierte Lösungen zur med. Dokumentation bei Wohnungslosen existieren, die eventuell als Grundlage verwendet werden könnten.
	\end{itemize}
\end{itemize}


\section{Motivation}

Die hier genannten Probleme stellen den Anlass für eine Entwicklung besserer Dokumentationssysteme im Zusammenhang mit der medizinischen Behandlung von wohnungslosen Menschen dar. Im weiteren Sinne kann diese Arbeit als einer der ersten Schritte in einem Projekt angesehen werden, das eine optimale Prozessgestaltung bei der Dokumentation für diese Personengruppen vorsieht und somit eine Entlastung des damit verbundenen Personals anstrebt.

Konkret sollen ärztliches Fachpersonal und Sozialarbeiter:innen Vorteile daraus ziehen können, indem Ihnen zur Entscheidungsfindung die relevanten Informationen zur benötigten Zeit am richtigen Ort zur Verfügung stehen. Dies ist nicht nur mit einer Zeiteinsparung verbunden, sondern bietet auch geringeren Spielraum für Fehlentscheidungen, insofern die patientenbezogenen Daten ausreichend vollständig und korrekt sind. Daraus folgt auch eine gewisse Sicherheit für die zu behandelnde Person, die mithilfe des ABS ihre Anonymität wahren kann und sich nicht der Belastung von unnötigen Doppeluntersuchungen stellen muss.

Eine erfolgreiche Projektdurchführung, in diesem Fall die Entwicklung besserer Dokumentationssysteme, hat eine detaillierte Wissensbasis bzw. eine Sammlung von Erfahrungsberichten als Voraussetzung. Unter der Betrachtung relevanter Informationen ist es möglich, etwaige Konfliktquellen und sonstige Schwierigkeiten im Voraus zu erkennen und in der Planung, Durchführung und Überwachung mit zu berücksichtigen.

Die Ergebnisse dieser Arbeit sind weiterhin als Ergebnis einer Systemanalyse und -bewertung zu verstehen, sodass darauf aufbauende Projekte einen Nutzen ziehen können. Dies schließt nicht nur das angesprochene Projekt von verbesserten Dokumentationssystemen bei der Behandlung von Wohnungslosen ein, sondern generelle Vorhaben, die das Wohl von Patienten und die Entlastung von medizinischem Personal als Ziel haben.

Die auszuwählende Methodik bei der Informationsbeschaffung kann bei ausreichend detaillierter Beschreibung als Beispiel für ähnliche Arbeiten genommen werden, die sich mit einer gleichen Problematik konfrontiert sehen. Die Auswahl konkreter Fachdatenbanken bzw. der entsprechenden Suchverfahren sollte dabei nachvollziehbar erklärt werden.

\newpage

\section{Zielsetzung}\label{sec:zielsetzung}

Um die oben genannten Probleme anzugehen, werden für diese Arbeit zu den einzelnen Punkten die folgenden Ziele definiert:

\begin{itemize}
	\item \textbf{Ziel Z1} zum Problem P1: Im Rahmen einer Systemanalyse und -bewertung des Leipziger Informationssystems zur medizinischen und sozialarbeitsbezogenen Dokumentation bei Wohnungslosen soll ein Analysebericht angefertigt werden, der zudem aktuelle Stärken und Schwächen aufzeigt.
	\item \textbf{Ziel Z2} zum Problem P2: Es soll eine Gegenüberstellung des Leipziger Informationssystems mit eventuell bereits bestehenden Ansätzen erstellt werden, in der deren Umsetzung bzw. Übertragbarkeit diskutiert wird.
\end{itemize}


\section{Aufgabenstellung}

Die zentralen Fragen, mit der sich diese Arbeit beschäftigt, sind \enquote{Wie läuft die derzeitige medizinische Dokumentation bei Wohnungslosen ab?} und \enquote{Wie lassen sich die Dokumentationssysteme in diesem Zusammenhang verbessern?}, wobei nochmals angemerkt werden muss, dass hierbei noch keine konkreten Systeme entwickelt oder implementiert werden sollen.

Zur Erreichung der gestellten Ziele, werden die anzugehenden Aufgaben wie folgt formuliert:

\begin{itemize}
	\item Aufgaben zu Z1:
	\begin{itemize}
		\item \textbf{Aufgabe A1.1:} Sammeln und Zusammenstellen von Informationen über das Leipziger Informationssystem der medizinischen Versorgung von Wohnungslosen
		\item \textbf{Aufgabe A1.2:} Gegenüberstellung der Stärken und Schwächen des modellierten Systems
		\item \textbf{Aufgabe A1.3:} Auflistung von Vorschlägen zur Verbesserung der Dokumentation bei Wohnungslosen
	\end{itemize}
	\item Aufgaben zu Z2:
	\begin{itemize}
		\item \textbf{Aufgabe A2.1:} Zielgerichtete Literaturrecherche über die Dokumentation bei Wohungslosen und über eventuell bereits erprobte Dokumentationssysteme
		\item \textbf{Aufgabe A2.2:} Gegenüberstellung der gesammelten Informationen mit dem Leipziger System sowie Diskussion der Praktikabilität in diesem Umfeld
	\end{itemize}
\end{itemize}

\section{Aufbau der Arbeit}

Dieses erste Kapitel soll zunächst eine Einführung in den Themenkomplex geben, wobei die bestehenden Probleme beschrieben und darauf aufbauend Ziele und Aufgaben formuliert werden, die den Rahmen dieser Arbeit darstellen. Die Grundlagen werden mit dem folgenden Kapitel behandelt. Hierbei sollen neben Begriffserklärungen auch Statistiken bzw. Publikationen hinzugezogen werden, die zum weiteren Verständnis erforderlich sind.

Kapitel 3 beinhaltet den aktuellen Stand der Informationen bzgl. der Dokumentation bei Wohnungslosen. Hierzu soll im anschließenden Kapitel ein Lösungsansatz beschrieben werden, welcher entsprechend der gesetzten Ziele und Aufgaben Methoden zur Informationsbeschaffung und -aufbereitung einschließt. Details zur Durchführung sowie die letztendlich zusammengestellten Ergebnisse bilden den Inhalt der Kapitel 5 und 6.

Das letzte Kapitel dient der kritischen Betrachtung der vorangestellten Ergebnisse sowie zur Diskussion im Zusammenhang mit den Aufgaben A1.3 und A2.2.